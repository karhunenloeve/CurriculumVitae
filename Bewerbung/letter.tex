\documentclass[11pt]{letter}
\usepackage[T1]{fontenc}
\usepackage[utf8]{inputenc}
\usepackage{lmodern}
\usepackage{geometry}
\usepackage{tgschola}
\geometry{a4paper, margin=25mm}

\signature{%
Luciano Melodia\\%
M.Sc.\, B.Sc.\, M.A.\, B.A.\\[0,2cm]%
Heckenweg 23\\%
91056 Erlangen\\%
Deutschland\\%
+49 175 3372526\\[0,2cm]%
\texttt{melodia.luciano@proton.me}}

\begin{document}

\begin{letter}{%
Diehl Defence GmbH & Co. KG, Röthenbach a. d. Pegnitz\\%
Fischbachstraße 16, 90552 Röthenbach an der Pegnitz\\%
}

\opening{Dear Diehl Defence Team,}

when I was at school, I was totally not into science. It was during the early years of secondary education that maths and physics were seen as the most challenging subjects, even though they were interesting. The only exception to this disenchantment was in computer science, where I actually did recognise a degree of aptitude. It was only later that I realised that building websites could actually be a really good way to make money. The projects from back then were beautiful to watch, but because I didn't really know how to use Github and version control, the source codes have since been lost. Thus, finishing the Abitur was the start of my academic journey.

The initial stage of my academic journey began at the University of Regensburg, located in my place of origin, with a focus on the humanities. My first degree was in German philology, with a major in Italian philology, information science and media informatics. In addition to my academic studies, I successfully completed a degree in Information Technology. My research interests lie predominantly in the domain of machine language processing. In pursuit of this specialisation, I engaged in a research project with Professor Rössler, focusing on information retrieval and punctuation. During my studies of Media Informatics and Information Science at university level, I was immediately admitted to the Master's degree in Information Science. I subsequently attended lectures on machine learning and optimisation. I completed my Master's degree with Professor Lang on the topic of deep learning for estimating the absorbed radiation dose of radioactive isotopes in cancer patients. The next stage of my educational development was Erlangen, where a number of renowned mathematicians and scientists held positions, including Max Noether, Felix Klein, Johann Radon and Walter Künneth. This period coincided with my involvement in a research project, which further contributed to my professional development. For a period of three years, I fulfilled the role of an assistant researcher at the Chair of Data Management at the Erlangen Department of Computer Science, under the supervision of Professor Lenz. I am pleased to inform you that I have successfully collaborated with Siemens Energy on the production of three publications. I am delighted to say that I was able to publish these as the sole author at conferences in the fields of mathematics and computer science. The subject area under discussion was always applied algebraic topology combined with artificial intelligence. The encouragement I received from mathematicians in Erlangen was instrumental in shaping my enduring passion for mathematics. It also led me to make the strategic decision to take a sabbatical from my professional pursuits in order to pursue academic studies in the field. As a student enrolled in the Bachelor's programme, I had the opportunity to attend a lecture on topology. It was at this juncture that I became certain of my forthcoming progression in my chosen discipline.

During my tenure at the Chair for Evolutionary Data Management, I successfully planned and executed a seminar on Topological Data Analysis, which was fully subscribed by fifteen students. Following their evaluation, the seminar received the exceptional rating of 1.14. In addition to my own coursework, I have provided tutoring support for Analysis II, III, Linear Algebra I, Topology and Topology & Applications on several occasions. The exercise groups I led received consistent praise and became a popular choice among students, which I found both motivating and gratifying. In order to maintain my proficiency in computer science, I elected to pursue a minor in the subject, with a particular focus on pattern recognition, computer vision and artificial intelligence. Much of the material was reminiscent of my previous studies, enabling me to apply a degree of prior knowledge and effort. I completed my bachelor's degree with a thesis on persistence modules and duality in persistent (co)homology under Professor Li.

For the duration of my master's programme, I remained at the university under the guidance of Professor Meusburger. I added a minor in Digital Humanities to reconnect with my earlier academic interests. In my studies, I specialised in the fields of algebra and geometry. I pursued courses in homological algebra, algebraic topology and tensor categories. I also focused on operator algebra, which deepened my understanding of measure theory and K-theory. These are relevant to persistent homology, given the persistence measure on the space of persistence diagrams. The focus of my master's thesis will be the Étale homology of ample groupoids.

I financed both my studies and my extracurricular pursuits independently, a fact of which I am proud. I come from a modest background, a situation that became even more austere after my father passed away, and I am therefore used to living within a limited budget. At the start of my studies, I took every opportunity to gain professional experience. I was fortunate enough to spend three years working as a weekend chef in a Bavarian restaurant that could accommodate up to 140 guests. This role provided me with hands-on experience in the demands and challenges of a professional kitchen environment. The industry also fostered enduring friendships, notably with Mr Böddecker, who welcomed me into his company, mb Support GmbH. There, I deployed neural networks on industrial projects and integrated them into the openVIVA C2 platform. I have held a number of part-time positions, which have allowed me to develop my skills in organisation, warehouse management and teamwork under pressure. These experiences have been both rewarding and instructive, and I have retained lasting contacts as a result. Concurrently, I pursued research assistant roles in linguistics with Professor Rössler, machine learning under Professor Lang, and subsequently in mathematics. I would like to express my gratitude for the support I received from the Oskar Karl Forster Foundation, which provided me with the necessary study equipment.

Before this account grows unwieldy, may I direct you to my curriculum vitae, which records my professional milestones in chronological order. Supporting documentation can, of course, be supplied upon request.

As part of my scientific work, I have authored a series of essays on mathematical subjects, including the Yoneda lemma, Fredholm operators and spectral sequences. These can be accessed via the links provided in my CV. My inaugural peer-reviewed paper was published at the international workshop Combinatorial Image Analysis in Novi Sad, Serbia. In this instance, Voronoi interpolation was employed for data augmentation. It was observed that the procedure is not reliably iterative; after repeated interpolations, the data's structure altered in a measurable way. By applying homological methods, we were able to identify the moment of this change with statistical rigour. My second major contribution was presented at the International Conference on Pattern Recognition in Milan, Italy. We analysed data lying on a torus, a natural choice because the time-series data, supplied by Siemens Energy, had been transformed via Takens embedding, thereby placing the Fourier transform of the sensor readings on a toroidal manifold. This analysis was conducted using the Künneth formula for persistent homology. From the resulting homology groups, we were able to infer the dimensionality required for data representation. This, in turn, enabled us to customise the neural network architecture. The approach is widely applicable, though it requires knowledge of the data's underlying manifold; in practice, we found that the dimension can often be estimated with reasonable accuracy even when the manifold is only conjectured. My latest publication, which is the result of my collaboration with Siemens Energy, was presented at the European Conference on Principles and Practice of Knowledge Discovery in Databases. Combining our previous results, we have classified sensor signals from four complete power plants in accordance with the plant-labelling system. The topologically faithful, Voronoi-augmented signals were fed into an appropriately dimensioned neural network — its hidden-layer size again gauged via the Künneth formula — together with Betti curves of dimension zero and one. For the first time, we have achieved over 50\% accuracy in identifying the type of sensor signal, as well as its physical provenance, and over 90\% accuracy in classifying the sensor type itself. The resulting systems have since been further refined and are now in operational use at Siemens Energy.

I made sure I kept up my physical training right up until the last minute – my parents always used to remind me that mens sana in corpore sano. When I was six, I joined SG Walhalla Regensburg and played football, judo, gymnastics and table tennis there until I was 14. I started weight-training when I was sixteen, but by eighteen I was back into gymnastics, entering competitions and street-art displays to test my skills. When I was twenty-four, I swapped the gym for Weng Chun Erlangen, where I studied Kung Fu until I got the green sash. That's when I discovered Muay Thai. I still really value this discipline. I trained really intensively at competitive level for six years and travelled to southern Thailand to test myself in the big arenas of Surat Thani with Lamai Muay Thai. My efforts paid off when I got black Pra Jiad armbands, which show that I'm a pro. There weren't many boxing facilities in Erlangen, so I switched back to weight training. I'm no longer a professional athlete, but I'd be happy to join a martial-arts club again.

For the post in question, I consider the following abilities to be essential: meticulous formal work, logical acuity and, above all, the ability to decompose complex problems into tractable sub-problems. I work efficiently and methodically under pressure, collaborate readily, and draw upon well-honed physical endurance and mental resilience. I take great satisfaction in motivating and supporting my colleagues and am always willing to impart my knowledge when required. In my opinion, academic study, industry employment and leisure pursuits form a single, holistic education. I am now aiming to apply this composite experience in a career. My background in teaching, research, writing and mathematics is, I believe, distinctive within the field. I would be delighted to bring these talents to your team and, in turn, to learn from you. I would therefore welcome the opportunity of an interview and a first personal meeting.

\closing{Yours faithfully,}

\end{letter}

\end{document}
