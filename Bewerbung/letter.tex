\documentclass[11pt]{letter}
\usepackage[T1]{fontenc}
\usepackage[utf8]{inputenc}
\usepackage{lmodern}
\usepackage{geometry}
\usepackage{tgschola}
\geometry{a4paper, margin=25mm}

\signature{%
Luciano Melodia\\%
M.Sc.\, B.Sc.\, M.A.\, B.A.\\[0,2cm]%
Heckenweg 23\\%
91056 Erlangen\\%
Deutschland\\%
+49 175 3372526\\[0,2cm]%
\texttt{melodia.luciano@proton.me}}

\begin{document}

\begin{letter}{%
Admissions Committee\\[2pt]%
\textit{(Institution Name)}\\%
\textit{(Department)}}

\opening{Dear Members of the Admissions Committee,}

During my school years, I exhibited no interest in scientific subjects whatsoever. It was during the early years of secondary education that the subject of mathematics and physics were perceived to be the most arduous, despite an inherent interest in the field. The sole exception to this disenchantment lay in the realm of computer science, in which a degree of aptitude was recognised. This was due to the realisation that the construction of websites held great potential for financial reward. The projects of that period were commendable, however, due to my lack of familiarity with Github and version control, the source codes have since been lost. The completion of the Abitur marked the commencement of the journey.

The initial stage of my academic journey began at the University of Regensburg, located in my place of origin, with a focus on the humanities. My first degree was in German philology, with a major in Italian philology, information science and media informatics. Alongside my studies, I successfully completed a degree in Information Technology. My research interests lie predominantly in the domain of machine language processing. In pursuit of this specialisation, I engaged in the research project with Professor Rössler, focusing on information retrieval and punctuation. During my studies of Media Informatics and Information Science at university level, I was immediately admitted to the Master's degree in Information Science. I subsequently attended lectures on machine learning and optimisation. I completed my Master's degree with Professor Lang on the topic of deep learning for estimating the absorbed radiation dose of radioactive isotopes in cancer patients. The subsequent stage of my academic journey was Erlangen, where a number of renowned mathematicians and scientists held positions, including Max Noether, Felix Klein, Johann Radon and Walter Künneth. This period coincided with my involvement in a research project, which further contributed to my professional development. For a period of three years, I fulfilled the role of an assistant researcher at the Chair of Data Management at the Erlangen Department of Computer Science, under the supervision of Professor Lenz. In collaboration with Siemens Energy, three publications were produced, which I was able to publish as the sole author in conferences in the fields of mathematics and computer science. The subject area under discussion was always applied algebraic topology. The profound encouragement of mathematicians in Erlangen served as a catalyst, fostering my enduring passion for mathematics and prompting a strategic decision to take a sabbatical from my professional pursuits in order to pursue academic studies in the field. As a student enrolled in the Bachelor's programme, I had the opportunity to attend a lecture on topology. It was at this point that I became certain of my future professional trajectory.

During my time at the Chair for Evolutionary Data Management I devised and delivered a seminar on Topological Data Analysis that was fully subscribed by fifteen students and, upon their appraisal, awarded the outstanding mark of 1.14. Alongside my own coursework I served several times as a tutor for Analysis II, III, Linear Algebra I, Topology, and Topology & Applications. My exercise groups were consistently praised and became something of a favourite among the students — an endorsement I found both motivating and gratifying. To keep my computational skills sharp I took Computer Science as a minor, concentrating on pattern recognition, computer vision, and artificial intelligence; much of the material felt happily familiar, allowing for a certain economy of effort. I completed my bachelor’s degree with a thesis on persistence modules and duality in persistent (co)homology under Professor Li.

For my master’s programme I remained at the university under the guidance of Professor Meusburger, adding a minor in Digital Humanities to reconnect with my earlier academic interests. Within mathematics I specialised in algebra and geometry, pursuing courses in homological algebra, algebraic topology, and tensor categories, while a second concentration in operator algebras deepened my understanding of measure theory and K-theory — both germane to persistent homology, given the persistence measure on the space of persistence diagrams. My master’s thesis will explore the Étale homology of groupoids.

I financed both my studies and my extracurricular pursuits entirely on my own, a fact of which I am quietly proud. I hail from modest beginnings — a situation rendered still more austere after my father’s passing — and have long been accustomed to frugal circumstances. Early in my academic career I seized every opportunity to earn my keep. By singular chance I spent three years as a weekend chef in a Bavarian restaurant that catered for up to 140 guests, learning the rigours of the professional kitchen. The hospitality trade also brought enduring friendships, notably with Mr Böddecker, who welcomed me into his company, mb Support, where I deployed neural networks on industrial projects and integrated them into the openVIVA C2 platform. Numerous other part-time posts sharpened my skills in organisation, warehouse management and teamwork under pressure — experiences that proved both rewarding and instructive, and from which I retain lasting contacts. In parallel, I embarked on research-assistant roles: first in linguistics with Professor Rössler, then in machine learning under Professor Lang, and later again in mathematics. My efforts were further supported by a study-equipment grant from the Oskar Karl Forster Foundation.

Before this account grows unwieldy, may I direct you to my curriculum vitae, which records my professional milestones in chronological order. Supporting documentation can, of course, be supplied upon request.

In the course of my scientific work I have written a series of essays on mathematical subjects — among them the Yoneda Lemma, Fredholm operators, and spectral sequences—which are available via the links in my CV. My first peer-reviewed paper appeared at the international workshop Combinatorial Image Analysis in Novi Sad, Serbia. There we employed Voronoi interpolation for data augmentation and observed that the procedure is not reliably iterative: after repeated interpolations the data’s structure alters in a measurable way. By applying homological methods we were able to pinpoint the moment of this change with statistical rigour.

My second major contribution was presented at the International Conference on Pattern Recognition in Milan, Italy. Drawing on the Künneth formula for persistent homology, we analysed data lying on a torus — a natural choice because the time-series data, supplied by Siemens Energy, had been transformed via Takens embedding, thereby placing the Fourier transform of the sensor readings on a toroidal manifold. From the resulting homology groups we directly inferred the dimensionality needed for data representation, which in turn allowed us to tailor the neural-network architecture. The approach is broadly applicable, though it requires knowledge of the data’s underlying manifold; in practice, we found that the dimension can often be estimated with reasonable accuracy even when the manifold is only conjectured.

My final publication arising from the collaboration with Siemens Energy appeared at the European Conference on Principles and Practice of Knowledge Discovery in Databases. Combining our previous results, we classified sensor signals from four complete power plants in accordance with the plant-labelling system. The topologically faithful, Voronoi-augmented signals were fed into an appropriately dimensioned neural network—its hidden-layer size again gauged via the Künneth formula — together with Betti curves of dimension zero and one. For the first time we achieved over fifty-per-cent accuracy in identifying not only the type of sensor signal but also its physical provenance, and over ninety-per-cent accuracy in classifying the sensor type itself. The resulting systems have since been further refined and are now in operational use at Siemens Energy.

I devoted the last slivers of spare time to physical training—my parents ever reminded me that \textit{mens sana in corpore sano}. At six years of age I joined SG Walhalla Regensburg, where I played football, judo, gymnastics and table tennis until the age of fourteen. Weight-training followed at sixteen; by eighteen I had gravitated back to gymnastics, entering freestyle competitions and street-art displays to test my skills. At twenty-four I exchanged the gymnasium for Weng Chun Erlangen, studying Kung Fu until I earned the green sash, and there discovered Muay Thai. This discipline still holds a special place for me. I trained intensively at competitive level for six years and travelled to southern Thailand, testing myself in the large arenas of Surat Thani with Lamai Muay Thai. My efforts were rewarded with black Pra Jiad armbands, signifying advanced prowess. A scarcity of facilities in Erlangen compelled me to end regular boxing sessions, and I returned to weight-training. My professional sporting career is now behind me, yet I should gladly join a martial-arts club again.

For the post in question I consider several abilities essential: meticulous formal work, logical acuity and, above all, the knack of decomposing complex problems into tractable sub-problems. I operate efficiently and methodically under time pressure, collaborate readily, and draw upon well-honed physical endurance and mental resilience. I take genuine pleasure in motivating and supporting colleagues and am always willing to teach when required. To me, academic study, industry employment and leisure pursuits form a single, holistic education. I now aim to apply this composite experience in a research career. My background in teaching, research, writing and mathematics is, I believe, distinctive within the field. I should be delighted to bring these talents to your team — and, in turn, to learn from you. I would therefore welcome the opportunity of an interview and a first personal meeting.

\closing{Yours faithfully,}

\end{letter}

\end{document}
