\documentclass[a4paper, 11pt]{article}
\usepackage[british]{babel}
\usepackage{classicthesis}
\usepackage{fontspec}

\setmainfont[Ligatures={Required,Common,Contextual,Rare,Historic,TeX},Numbers=OldStyle,RawFeature={+ss05,+dlig,+hlig,+calt,+liga},ItalicFeatures={RawFeature={+cv04,+clig,+swsh,+calt,+liga,+hlig,+ss05},CharacterVariant=5:0}]{EB Garamond}
\usepackage{amsmath}
\usepackage{fontawesome}
\usepackage{geometry}
\usepackage{marginnote}
\usepackage{sectsty}
\usepackage{academicons}
\usepackage{soul}
\usepackage{hanging}
\usepackage{enumitem}
\usepackage{etaremune}
\usepackage{hyperref}
\geometry{a4paper, textwidth=15cm, textheight=25.0cm, marginparwidth=1.5cm}
\setlength{\parindent}{0pt}
\setlength{\marginparsep}{10pt}
\setlength{\skip\footins}{0.5cm}

\setlist[enumerate]{nosep}
\newcommand{\rd}{\textsuperscript{\textup{rd}}\xspace}
\newcommand{\nd}{\textsuperscript{\textup{nd}}\xspace}
\renewcommand{\th}{\textsuperscript{\textup{th}}\xspace}
\newcommand{\authorequal}{\kern-0.1em\textsuperscript{\dagger}}
\renewcommand*{\raggedleftmarginnote}{}
\reversemarginpar
\newcommand{\years}[1]{\marginnote{\scriptsize #1}}

\allsectionsfont{\mdseries\bfseries}
\setlist[itemize]{noitemsep, topsep=0pt}

%%%%%%%%%%%%%%%%%%%%%%%%%%%%%%%%%%%%%%%%%%%%%%%%%%%%%%%%%%%%%%%%%%%%%%%%
% Main document
%%%%%%%%%%%%%%%%%%%%%%%%%%%%%%%%%%%%%%%%%%%%%%%%%%%%%%%%%%%%%%%%%%%%%%%%

\begin{document}
	{\Huge \emph{Luciano Melodia}}\\[0.1cm] \emph{Curriculum vitae}\\ \emph{Last
	update on \today}.

	\begin{flushleft}
		\scriptsize
		\begin{minipage}{0.3\textwidth}
			{\footnotesize \faEnvelope} \hspace{0.1cm} \href{mailto:luciano.melodia@fau.de}{luciano.melodia@fau.de}\\[0.05cm]
			{\footnotesize \faGithub} \hspace{0.15cm}
			\href{https://github.com/karhunenloeve}{karhunenloeve}\\[0.05cm] {\footnotesize \aiarXiv}
			\hspace{0.1cm} \href{https://arxiv.org/a/melodia_l_1}{melodia\_l\_1}
		\end{minipage}
		\begin{minipage}{0.4\textwidth}
			{\footnotesize \aiOrcid} \hspace{0.1cm} \href{https://orcid.org/0000-0002-7584-7287}{0000-0002-7584-7287}
 \\[0.05cm]
			{\footnotesize \faMapO} \hspace{0.1cm} Regensburg, Germany
		\end{minipage}
	\end{flushleft}

	%%%%%%%%%%%%%%%%%%%%%%%%%%%%%%%%%%%%%%%%%%%%%%%%%%%%%%%%%%%%%%%%%%%%%%%%
	\section*{Professions}
	%%%%%%%%%%%%%%%%%%%%%%%%%%%%%%%%%%%%%%%%%%%%%%%%%%%%%%%%%%%%%%%%%%%%%%%%
	\years{2023--25} Student Assistant at the chairs Algebra and Geometry, \\ Representation
	Theory and Operator Algebras, \\ Applied Analysis and \\ Applied Mathematics at
	Friedrich-Alexander Universität Erlangen-Nürnberg.
	\begin{itemize}
		\item Tutor in \emph{Mathematics for Engineers A4 (stochastic)}.

		\item Tutor in \emph{Analysis 2}.

		\item Tutor in \emph{Analysis 3}.

		\item Tutor in \emph{Topology and Applications}.

		\item Tutor in \emph{Topology}.

		\item Tutor in \emph{Linear Algebra 1}.

		\item Lecture representation of Prof. Li on the \emph{Tietze Extension Theorem}.

		\item Lecturer for mathematical proof sessions.

		\item Lecturer for exercise sessions.

		\item Supervision and correction of written exams.
	\end{itemize}
	\years{2025} Lecturer in Mathematics at Paukkammer Erlangen.\\
	\years{2024--25} Tutor in Mathematics:
	\begin{itemize}
		\item Computer Science, 1 student, Friedrich-Alexander Universität Erlangen-Nürnberg.

		\item Chemical and Biological Engineering, 1 student, Friedrich-Alexander Universität Erlangen-Nürnberg.

		\item Physics, 1 student, Friedrich-Alexander Universität Erlangen-Nürnberg.

		\item Abitur, 13 students, Bavarian Gymnasium.

		\item Secondary school, 2 students, Bavarian Mittelschule.

		\item Secondary school, 3 students, Bavarian Realschule.

		\item Elementary school, 1 student, Bavarian Grundschule.
	\end{itemize}
	\years{2021--22} Werkstudent at Corscience GmbH \& Co. KG, Erlangen.
	\begin{itemize}
		\item Deep convolutional networks trained on multiple GPUs for automatic detection of calibration spikes in ECG data. Achieved an accuracy of over $\approx 0.99$ on ten-fold cross validation with a data set of about $10^6$ real world samples tested with $5 \sigma$ significance, which is state of the art.

		\item Residual networks for detection of ECG curves in documents. Achieved an intersection over union of $\approx 0.98$ on ten-fold cross validation with a data set of about $10^7$ artificially enlarged using generative neural networks tested with $3\sigma$ significance, which is state of the art.

		\item Image segmentation using matrix factorisation techniques to isolate ECG curves. Achieved an intersection over union of $\approx 0.99$ tested with $6 \sigma$ significance, which is state of the art.
	\end{itemize}
	\years{2019--21} Researcher at Siemens Energy AG, Erlangen.
	\begin{itemize}
		\item Programming with CUDA v.11.0, Tensorflow 2.4, CuDNN v.8.0.4. in Python v.3.8 and v.3.9.

		\item Operating systems: Ubuntu 20.04, Solus 4, Archlinux 5.11, Windows 11. 

		\item Research published:
		\begin{itemize}
			\item \emph{Luciano Melodia} and \ul{Richard Lenz}: Homological Time Series Analysis of Sensor Signals from Power Plants. Machine Learning for Irregular Time Series. Machine Learning and Principles and Practice of Knowledge Discovery in Databases. In \ul{Michael Kamp}, \ul{Irena Koprinska}, \ul{Adrien Bibal} et al. (ed.): Communications in Computer and Information Science. Springer Nature, Switzerland.
			\item \emph{Luciano Melodia} and \ul{Richard Lenz}: Estimate of the Neural Network Dimension Using Algebraic Topology and Lie Theory. Image Mining. Theory and Applications VII. Pattern Recognition and Information Forensics. In \ul{Alberto Del Bimbo}, \ul{Rita Cucchiara}, \ul{Stan Sciaroff} et al. (ed.): Lecture Notes in Computer Science. Springer Nature, Switzerland.
			\item \emph{Luciano Melodia} and \ul{Richard Lenz}: Persistent Homology as a Stopping Criterion for Voronoi Interpolation. Proceedings of the International Workshop on Combinatorial Image Analysis. In \ul{Tibor Lukić}, \ul{Reneta Barneva}, \ul{Valentin Brimkov} et al. (ed.): Lecture Notes in Computer Science. Springer, Cham.
		\end{itemize}
	\end{itemize}
	\years{2018--21} Researcher Chair of Evolutionary Information Systems (Computer Science 6), Friedrich-Alexander Universität Erlangen-Nürnberg.
	\begin{itemize}
		\item Correction of written exams and assistance in oral exams.

		\item Preparation and execution of electronic exams.

		\item Participation in the Data Science program.

		\item Planning and implementation of the
			\begin{itemize}
				\item lecture \emph{Knowledge Discovery in Databases}.

				\item seminar \emph{Persistent Homology in Data Analytics}.

				\item seminar \emph{Topological Data Analysis}, (score 1.14).

				\item seminar \emph{New Technologies in Data Management}.

				\item exercise lessons in \emph{Process Oriented Information Systems}, (score 1.18).

				\item exercise lessons in \emph{Computer Science for Engineers}.

				\item exercise lessons in \emph{Conceptual Modeling}.
			\end{itemize}

		\item Supervision of thesis:
			\begin{itemize}[noitemsep, leftmargin=*]
				\item Bachelor of Science:
					\begin{itemize}
						\item B.Sc. Computer Science, Hahn (2021): \emph{Classification of Sensor
							Signals from Power Plants}.

						\item B.Sc. Computer Science, Schäfer (2021): \emph{Learning Validation
							Models from Sensors of a Power Plant}.
					\end{itemize}

				\item Master of Science:
					\begin{itemize}
						\item M.Sc. Computer Science, Sauerhammer (2021): \emph{A Classification
							Dashboard for Sensor Signals from Power Plants}.

						\item M.Sc. Mechanical Engineering, Seidel (2020): \emph{Classification of
							Microbes using Time Series Gas Sensor Array Data}.

						\item M.Sc. Medical Engineering, Siddiqui (2020): \emph{Extraction of
							Fetal and Maternal Heartbeats from ECG Signals}.
					\end{itemize}
			\end{itemize}
	\end{itemize}
	\years{2015--18} Data Scientist at mb Support GmbH, Regensburg.
	\begin{itemize}
		\item Construction of a fully automated document scanning pipeline for industrial use for massive digitization of paperpiles (up to $6 \times 10^7$ documents). Engineering of the scanning street, OCR recognition with a tested benchmark result of CER $1\%$ using Googles Cloud Vision APIs and self trained recurrent neural networks. Integration of the system with a full user interface including the software ergonomic engineering for \emph{openviva C2}.

		\item Software ergonomic engineering and integration of the \emph{Asterisk}s telecommunication API into \emph{openviva C2}. About $5 \times 10^3$ lines of code in PL/SQL and Python.

		\item Statistical data and market analysis using large deep neural networks, large deep convolutional networks and regression techniques.
	\end{itemize}
	\years{2013--15} Research assistant Chair of German Linguistics, Universität
	Regensburg.
	\begin{itemize}
		\item Examination correction, correction of books and texts.

		\item Implementation of the \emph{punc.space} web platform with a custom search engine written in Javascript for realtime online usage with up to $10^3$ lines of code.

		\item Maintenance of the university website.

		\item Organization and conduct of conferences.
	\end{itemize}
	\years{2012--15} Chef in event gastronomy at Apostelkeller, Regensburg.
	\begin{itemize}
		\item Cooking according to a fixed menu for up to 140 guests.

		\item Waitressing and stock management.
	\end{itemize}
	\years{2012--15} Staff-based services at Trademarketing Service GmbH,
	Salzgitter.
	\begin{itemize}
		\item Goods management and ordering.

		\item Goods receipt.
	\end{itemize}
	\years{2012--14} Translator at Anatol GmbH \& Co. KG, Regensburg.
	\begin{itemize}
		\item Italian -- German translation.

		\item Polish -- German translation.

		\item English -- German translation.
	\end{itemize}
	\years{2010} Volunteer at Alten- und Pflegeheim St. Josef, Regensburg.

	\newpage
	%%%%%%%%%%%%%%%%%%%%%%%%%%%%%%%%%%%%%%%%%%%%%%%%%%%%%%%%%%%%%%%%%%%%%%%%
	\section*{Academic Work}
	%%%%%%%%%%%%%%%%%%%%%%%%%%%%%%%%%%%%%%%%%%%%%%%%%%%%%%%%%%%%%%%%%%%%%%%%
	\years{Teaching}
	\vspace{-2pt}
	\begin{itemize}[noitemsep, leftmargin=*]
		\item Department of Mathematics, \\ Friedrich-Alexander Universität Erlangen-Nürnberg
			\begin{itemize}
				\item 2025 Exercises in \emph{Mathematics for Engineers A4}.

				\item 2025 Exercises in \emph{Analysis 2}.

				\item 2025 Exercises in \emph{Analysis 3}.

				\item 2024 Exercises in \emph{Topology and Applications}.

				\item 2024 Exercises in \emph{Linear Algebra 1}.

				\item 2023 Exercises in \emph{Topology}.
			\end{itemize}

		\item Department of Computer Science, \\ Friedrich-Alexander Universität
			Erlangen-Nürnberg
			\begin{itemize}
				\item 2021 Lecture on \emph{Knowledge Discovery in Databases}.

				\item 2021 Exercises in \emph{Process Oriented Information Systems}.

				\item 2021 Seminar on \emph{New Technologies in Data Management}.

				\item 2021 Exercises in \emph{Computer Science for Engineers}.

				\item 2020 Seminar on \emph{Persistent Homology in Data Analytics}.

				\item 2020 Seminar on \emph{Topological Data Analysis}.

				\item 2020 Exercises in \emph{Process Oriented Information Systems}.

				\item 2020 Exercises in \emph{Computer Science for Engineers}.

				\item 2020 Seminar on \emph{New Technologies in Data Management}.

				\item 2019 Exercises in \emph{Computer Science for Engineers}.

				\item 2019 Exercises in \emph{Process Oriented Information Systems}.

				\item 2019 Seminar on \emph{New Technologies in Data Management}.

				\item 2018 Exercises in \emph{Computer Science for Engineers}.

				\item 2018 Seminar on \emph{New Technologies in Data Management}.

				\item 2018 Exercises in \emph{Conceptual Modeling}.
			\end{itemize}
	\end{itemize}
	\vspace{10pt}

	\years{Conferences}
	\vspace{-10pt}
	\begin{itemize}[noitemsep, leftmargin=*]
		\item Mathematics:
			\begin{itemize}[noitemsep, leftmargin=*]
				\item 2020 Topological Data Analysis and Beyond.

				\item 2020 International Workshop on Combinatorial Image Analysis.

				\item 2020 International Conference on Practical Mathematical Discourse.
			\end{itemize}

		\item Computer Science:
			\begin{itemize}[noitemsep, leftmargin=*]
				\item 2024 Learning on Graphs.

				\item 2023 Learning on Graphs.

				\item 2023 International Conference on Advances in Databases, Knowledge,
					and Data Applications.

				\item 2022 Learning on Graphs.

				\item 2022 International Conference on Learning Representations.

				\item 2021 Machine Learning for Irregular Time Series.

				\item 2021 International Conference on Pattern Recognition.

				\item 2020 European Conference on Machine Learning and Principles and Practice
					of Knowledge Discovery in Databases.

				\item 2019 European Conference on Machine Learning and Principles and Practice
					of Knowledge Discovery in Databases.

				\item 2019 Symposium on Principles of Database Systems.
			\end{itemize}

		\item Linguistics:
			\begin{itemize}[noitemsep, leftmargin=*]
				\item 2017 Kolloquium zum Sprachmanagement.

				\item 2013 Destandardisierung und Standardvarietät.
			\end{itemize}
	\end{itemize}
	\vspace{10pt}

	\years{Service}
	\vspace{-10pt}
	\begin{itemize}[noitemsep, leftmargin=*]
		\item Awards:
			\begin{itemize}[noitemsep, leftmargin=*]
				\item 2024 Top reviewer award at Learning on Graphs.

				\item 2024 Oskar-Karl-Forster scholarship fellow.
			\end{itemize}

		\item Reviewer:
			\begin{itemize}[noitemsep, leftmargin=*]
				\item 2024 Learning on Graphs.

				\item 2024 International Conference on Advances in Databases, Knowledge,
					and Data Applications.

				\item 2023 Learning on Graphs.

				\item 2023 International Conference on Advances in Databases, Knowledge,
					and Data Applications.

				\item 2022 Learning on Graphs.

				\item 2022 International Conference on Advances in Databases, Knowledge,
					and Data Applications.

				\item 2022 Reviewer for the Workshop Geometrical and Topological Representation
					Learning at International Conference on Learning Representations.

				\item 2021 Reviewer for the Workshop Topological Data Analysis and Beyond
					at Neural Information Processing Systems.

				\item 2021 International Conference on Advances in Databases, Knowledge,
					and Data Applications.

				\item 2020 International Conference on Advances in Databases, Knowledge,
					and Data Applications.
			\end{itemize}

		\item Memberships:
			\begin{itemize}[noitemsep, leftmargin=*]
				\item 2019 -- 20 Member of the Gesellschaft für Informatik e.V.

				\item 2017 -- 18 Member of the Computational Intelligence and Machine
					Learning Group, Universität Regensburg.
			\end{itemize}

		\item 2024 Student Representative for the Department of Mathematics at Friedrich-Alexander
			Universität Erlangen-Nürnberg.

		\item 2016 Student Representative for the Department of Language, Literature
			and Cultural Sciences at Universität Regensburg.
	\end{itemize}
	\vspace{10pt}

	\newpage
	%%%%%%%%%%%%%%%%%%%%%%%%%%%%%%%%%%%%%%%%%%%%%%%%%%%%%%%%%%%%%%%%%%%%%%%%
	\section*{Papers}
	%%%%%%%%%%%%%%%%%%%%%%%%%%%%%%%%%%%%%%%%%%%%%%%%%%%%%%%%%%%%%%%%%%%%%%%%
	\begin{hangparas}
		{1em}{1} \years{2021 \href{https://doi.org/10.1007/978-3-030-93736-2_22}{\aiDoi} \href{https://arxiv.org/pdf/2106.02493}{\aiarXiv}}
		\emph{Luciano Melodia} and \ul{Richard Lenz}: Homological Time
		Series Analysis of Sensor Signals from Power Plants. Machine Learning for Irregular
		Time Series. Machine Learning and Principles and Practice of Knowledge
		Discovery in Databases. In \ul{Michael Kamp}, \ul{Irena Koprinska},
		\ul{Adrien Bibal} et al. (ed.): Communications in Computer and Information
		Science. Springer Nature, Switzerland.
	\end{hangparas}
	%
	\begin{hangparas}
		{1em}{1} \years{2021 \href{https://link.springer.com/10.1007/978-3-030-68821-9_2}{\aiDoi} \href{https://arxiv.org/pdf/2004.02881}{\aiarXiv}}
		\emph{Luciano Melodia} and \ul{Richard Lenz}: Estimate of the
		Neural Network Dimension Using Algebraic Topology and Lie Theory. Image Mining.
		Theory and Applications VII. Pattern Recognition and Information Forensics.
		In \ul{Alberto Del Bimbo}, \ul{Rita Cucchiara}, \ul{Stan Sciaroff}
		et al. (ed.): Lecture Notes in Computer Science. Springer Nature,
		Switzerland.
	\end{hangparas}
	%
	\begin{hangparas}
		{1em}{1} \years{2020 \href{https://link.springer.com/chapter/10.1007\%2F978-3-030-51002-2_3}{\aiDoi} \href{https://arxiv.org/pdf/1911.02922}{\aiarXiv}}
		\emph{Luciano Melodia} and \ul{Richard Lenz}: Persistent Homology
		as a Stopping Criterion for Voronoi Interpolation. Proceedings of the International
		Workshop on Combinatorial Image Analysis. In \ul{Tibor Lukić},
		\ul{Reneta Barneva}, \ul{Valentin Brimkov} et al. (ed.):
		Lecture Notes in Computer Science. Springer, Cham.
	\end{hangparas}

	\begin{hangparas}
		{1em}{1} \years{2015 \href{https://www.logos-verlag.de/cgi-bin/engbuchmid?isbn=3808&lng=deu}{\aiDoi} \href{https://ling.auf.net/lingbuzz/004798}{\aiarXiv}}
		\emph{Luciano Melodia}: On the Use of the Paradigm \emph{brauchen} with
		and without \emph{zu} with Infinitives. In \ul{Kateřina Šichová},
		\ul{Reinhard Krapp}, \ul{Paul Rössler} et al. (ed.): Standard
		Varieties of German -- Case Studies from Social Practice, Logos, Berlin.
	\end{hangparas}
	%
	\vspace{0.5cm}

	%%%%%%%%%%%%%%%%%%%%%%%%%%%%%%%%%%%%%%%%%%%%%%%%%%%%%%%%%%%%%%%%%%%%%%%%
	\section*{Theses}
	%%%%%%%%%%%%%%%%%%%%%%%%%%%%%%%%%%%%%%%%%%%%%%%%%%%%%%%%%%%%%%%%%%%%%%%%
	\begin{hangparas}
		{1em}{1} \years{2025 \href{}{\faFilePdfO} \href{}{\aiarXiv}}
		\emph{Luciano Melodia}: Universal Coefficients for Ètale Groupoid Homology. Library of
		the \ul{Friedrich-Alexander Universität Erlangen-Nürnberg}.
		\ul{Master thesis} in Mathematics.
	\end{hangparas}
	%
	\begin{hangparas}
		{1em}{1} \years{2024 \href{https://karhunenloeve.github.io/BscMath/main.pdf}{\faFilePdfO} \href{https://arxiv.org/pdf/2410.08323}{\aiarXiv}}
		\emph{Luciano Melodia}: Algebraic and Topological Persistence. Library of
		the \ul{Friedrich-Alexander Universität Erlangen-Nürnberg}.
		\ul{Bachelor thesis} in Mathematics.
	\end{hangparas}
	%
	\begin{hangparas}
		{1em}{1} \years{2018 \href{https://arxiv.org/pdf/1805.09108}{\faFilePdfO} \href{https://arxiv.org/pdf/1805.09108}{\aiarXiv}}
		\emph{Luciano Melodia}: Deep Learning Estimation of Absorbed Radiation Dose
		for Nuclear Medicine Diagnostics. Library of the \ul{Universität Regensburg},
		\ul{Master Thesis} in Information Science.
	\end{hangparas}
	%
	\begin{hangparas}
		{1em}{1} \years{2015} \emph{Luciano Melodia}: Entwicklung einer Interpunktionsplattform mit linguistischen Moduln für das Information Retrieval. Library of the \ul{Universität Regensburg}, \ul{Bachelor Thesis} in German Philology.
	\end{hangparas}
	%
	\vspace{0.5cm}

	%%%%%%%%%%%%%%%%%%%%%%%%%%%%%%%%%%%%%%%%%%%%%%%%%%%%%%%%%%%%%%%%%%%%%%%%
	\section*{Notes}
	%%%%%%%%%%%%%%%%%%%%%%%%%%%%%%%%%%%%%%%%%%%%%%%%%%%%%%%%%%%%%%%%%%%%%%%%
	\begin{hangparas}
		{1em}{1} \years{2025 \href{https://karhunenloeve.github.io/Mathematical-Notes-2/chainhomotopies.pdf}{\faFilePdfO}}
		\emph{Luciano Melodia}: Kettenhomotopien. Graduate Seminar on Homological Algebra in Representation Theory, \ul{Friedrich-Alexander Universität Erlangen-Nürnberg}.
	\end{hangparas}
	%
	\begin{hangparas}
		{1em}{1} \years{2025 \href{https://karhunenloeve.github.io/Mathematical-Notes-2/Hochschild__Ko_homologie.pdf}{\faFilePdfO}}
		\emph{Luciano Melodia}: Hochschild(ko)homologie. Graduate Seminar on Homological Algebra in Representation Theory, \ul{Friedrich-Alexander Universität Erlangen-Nürnberg}.
	\end{hangparas}
	%
	\begin{hangparas}
		{1em}{1} \years{2025 \href{https://karhunenloeve.github.io/SpecSeq/main.pdf}{\faFilePdfO}}
		\emph{Luciano Melodia}: Spektrale Sequenzen - Leray-Serre spektrale Sequenz. Graduate Seminar on Spectral Theory in Mathematical Physics, \ul{Friedrich-Alexander Universität Erlangen-Nürnberg}.
	\end{hangparas}
	%
	\begin{hangparas}
		{1em}{1} \years{2024 \href{https://karhunenloeve.github.io/FunkanaFredholm/main.pdf}{\faFilePdfO}}
		\emph{Luciano Melodia}: Beschränkte Fremdholmoperatoren und deren Fremdholmindex
		auf separablen Hilberträumen. Graduate Seminar on Spectral Flow in
		Functional Analysis, \ul{Friedrich-Alexander Universität Erlangen-Nürnberg}.
	\end{hangparas}
	%
	\begin{hangparas}
		{1em}{1} \years{2023 \href{https://karhunenloeve.github.io/TopoHom/main.pdf}{\faFilePdfO}}
		\emph{Luciano Melodia}: Notes on Simplicial and Singular Homology.
		Graduate Seminar on Topics in Topology. \ul{Friedrich-Alexander Universität Erlangen-Nürnberg}.
	\end{hangparas}
	%
	\begin{hangparas}
		{1em}{1} \years{2022 \href{https://karhunenloeve.github.io/TopoSheaf/main.pdf}{\faFilePdfO}}
		\emph{Luciano Melodia}: Natürliche Transformationen, Äquivalenzen von
		Kategorien, darstellbare Funktoren und das Lemma von Yoneda. Undergraduate
		Seminar on Sheaf Theory. \ul{Friedrich-Alexander Universität Erlangen-Nürnberg}.
	\end{hangparas}
	%
	\vspace{0.5cm}

	\newpage
	%%%%%%%%%%%%%%%%%%%%%%%%%%%%%%%%%%%%%%%%%%%%%%%%%%%%%%%%%%%%%%%%%%%%%%%%
	\section*{Education}
	%%%%%%%%%%%%%%%%%%%%%%%%%%%%%%%%%%%%%%%%%%%%%%%%%%%%%%%%%%%%%%%%%%%%%%%%
	\begin{itemize}[noitemsep, leftmargin=*]
		\item \years{2024 -- 26} Master of Science in Mathematics, Friedrich-Alexander
			Universität.
			\begin{itemize}[noitemsep, leftmargin=*]
				\item Topic: Groupoid Homology.
				\item Minor: Digital Humanities.
			\end{itemize}

		\item \years{2021 -- 24} Bachelor of Science in Mathematics, Friedrich-Alexander
			Universität.
			\begin{itemize}[noitemsep, leftmargin=*]
				\item Topic: Algebraic and Topological Persistence (grade 1.0).

				\item Minor: Computer Science.
			\end{itemize}

		\item \years{2015 -- 18} Master of Arts in Information Science, Universität Regensburg.
			\begin{itemize}[noitemsep, leftmargin=*]
				\item Topic: Deep Learning for Radiation Dose Calculation (grade 1.3).
			\end{itemize}

		\item \years{2012 -- 15} Bachelor of Arts in German Philology, Universität Regensburg.
			\begin{itemize}[noitemsep, leftmargin=*]
				\item Topic: Information Retrieval and Punctuation (grade 1.7).

				\item Majors: Italian Philology, Information Science, Media Informatics.
			\end{itemize}

		\item \years{2012 -- 13} Studienbegleitende IT-Ausbildung (grade 1.7), Rechenzentrum Universität Regensburg.

		\item \years{2012} Abitur, Albertus-Magnus-Gymnasium, Regensburg.
	\end{itemize}

	%%%%%%%%%%%%%%%%%%%%%%%%%%%%%%%%%%%%%%%%%%%%%%%%%%%%%%%%%%%%%%%%%%%%%%%%
	\section*{Certificates}
	%%%%%%%%%%%%%%%%%%%%%%%%%%%%%%%%%%%%%%%%%%%%%%%%%%%%%%%%%%%%%%%%%%%%%%%%
	\begin{itemize}[noitemsep, leftmargin=*]
		\item {Udemy:}
			\begin{itemize}[itemsep=0pt, leftmargin=15pt]
				\item \years{2021} The Rust Programming Language.

				\item \years{2020} The Python Mega Course: Build 10 Real World Applications.
			\end{itemize}

		\item Imperial College London:
			\begin{itemize}[itemsep=0pt, leftmargin=15pt]
				\item \years{2018} Mathematics for ML - Multivariate Calculus.

				\item \years{2018} Mathematics for ML - Linear Algebra.
			\end{itemize}

		\item Shanghai Jiao Tong University:
			\begin{itemize}[itemsep=0pt, leftmargin=15pt]
				\item \years{2018} Discrete Mathematics.
			\end{itemize}

		\item Wesleyan University:
			\begin{itemize}[itemsep=0pt, leftmargin=15pt]
				\item \years{2018} Introduction to Complex Analysis.
			\end{itemize}

		\item Coursera:
			\begin{itemize}[itemsep=0pt, leftmargin=15pt]
				\item \years{2018} Exploratory Data Analysis.

				\item \years{2018} Intermediate R - Practice Course.

				\item \years{2018} Intermediate R.

				\item \years{2018} Introduction to R.

				\item \years{2018} Supervised Learning in R - Regression.

				\item \years{2018} Supervised Learning in R - Classification.

				\item \years{2018} Text Mining - Bag of Words.

				\item \years{2018} Deep Learning in Python.

				\item \years{2018} Introduction to Machine Learning.

				\item \years{2018} Intro to Python for Data Science.

				\item \years{2018} Machine Learning Toolbox.

				\item \years{2018} Credit Risk Modeling in R.

				\item \years{2018} Data Visualization in R.

				\item \years{2018} Data Visualization with ggplot2 II.

				\item \years{2018} Data Visualization with ggplot2 I.
			\end{itemize}

		\item Others:
			\begin{itemize}[itemsep=0pt, leftmargin=15pt]
				\item \years{2016} Beer Sommelièr, Sperber Bräu.
			\end{itemize}
	\end{itemize}

	%%%%%%%%%%%%%%%%%%%%%%%%%%%%%%%%%%%%%%%%%%%%%%%%%%%%%%%%%%%%%%%%%%%%%%%%
	\section*{Interests}
	%%%%%%%%%%%%%%%%%%%%%%%%%%%%%%%%%%%%%%%%%%%%%%%%%%%%%%%%%%%%%%%%%%%%%%%%
	\years{Coding} Python, JavaScript.\\
	\years{Software} GUDHI, Dionysus, Keras.\\
	\years{Languages} German, English (C2), Italian (C2), Polish (B2), Spanish (A2).\\
	\years{Hobbies} Cooking, Reading.\\
	\years{Sports} Functional training.

	%%%%%%%%%%%%%%%%%%%%%%%%%%%%%%%%%%%%%%%%%%%%%%%%%%%%%%%%%%%%%%%%%%%%%%%%
	\section*{References}
	%%%%%%%%%%%%%%%%%%%%%%%%%%%%%%%%%%%%%%%%%%%%%%%%%%%%%%%%%%%%%%%%%%%%%%%%
	\years{Prof. Dr.} Kang Li\\ Department of Mathematics\\ Friedrich-Alexander
	University Erlangen-Nürnberg\\ Professor for Representation Theory and Operator
	Algebras\\ \years{\hspace{1cm}\faEnvelope} \url{kang.li@fau.de}\\ \years{\hspace{1cm}\faPhone}
	+49 9131 85-67060\\

	\years{Prof. Dr.} Richard Lenz\\ Department of Computer Science\\ Friedrich-Alexander
	University Erlangen-Nürnberg\\ Professor for Evolutionary Data Management\\ \years{\hspace{1cm}\faEnvelope}
	\url{richard.lenz@fau.de}\\ \years{\hspace{1cm}\faPhone} +49 9131 85-27899\\

	\years{Prof. Dr. em.} Elmar Lang\\ Department of Biophysics\\ Universität Regensburg \\ Professor
	for Computational Intelligence\\ \years{\hspace{1cm}\faEnvelope}
	\url{elmar.w.lang@ur.de}\\

	\years{Prof. Dr.} Paul Rössler\\ Department of German Philology\\ Universität Regensburg \\ Professor
	for German Linguistics\\ \years{\hspace{1cm}\faEnvelope} \url{paul.roessler@ur.de}\\
	\years{\hspace{1cm}\faPhone} +49 941 943-3444\\
\end{document}