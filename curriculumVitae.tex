\documentclass[a4paper,11pt]{article}

\usepackage[british]{babel}
\usepackage{fontspec}
\usepackage{fontawesome}
\usepackage{geometry}
\usepackage{marginnote}
\usepackage{scrextend}
\usepackage{sectsty}
\usepackage{microtype}
\usepackage[usenames,dvipsnames]{color}
\usepackage{academicons}
\usepackage{hanging}
\usepackage{enumitem}
\usepackage{etaremune}
\usepackage{tikz}
\usepackage{classicthesis}
\usepackage[export]{adjustbox}
\geometry{
  a4paper,
  textwidth      = 14.0cm,
  textheight     = 25.0cm,
  marginparwidth =  2.5cm
}
\defaultfontfeatures{Ligatures=TeX}
\setlength{\parindent}{0pt}
\setlength{\marginparsep}{10pt}

\setlength{\parindent}{0pt}
\setlength{\skip\footins}{0.5cm}

\definecolor{amber}{rgb}{1.0, 0.75, 0.0}
\definecolor{bittersweet}{rgb}{1.0, 0.44, 0.37}
\definecolor{darkcyan}{rgb}{0.0, 0.55, 0.55}
\definecolor{darkmagenta}{rgb}{0.55, 0.0, 0.55}
\definecolor{denim}{rgb}{0.08, 0.38, 0.74}

\setlist[enumerate]{nosep}
\NewDocumentCommand{\statcirc}{ O{#2} m }{%
    \begin{tikzpicture}[every node/.style={inner sep=0,outer sep=-1}]
    \fill[#2] (0,0) circle (0.55ex);
    \fill[#1] (0,0) -- (180:0.55ex) arc (180:0:0.55ex) -- cycle;
    \end{tikzpicture}
}
\newcommand  {\rd}{\textsuperscript{\textup{rd}}\xspace}
\newcommand  {\nd}{\textsuperscript{\textup{nd}}\xspace}
\renewcommand{\th}{\textsuperscript{\textup{th}}\xspace}
\newcommand{\authorequal}{\kern-0.1em\textsuperscript{\dagger}}
\renewcommand*{\raggedleftmarginnote}{}
\reversemarginpar
\newcommand{\years}[1]{\marginnote{\scriptsize #1}}

\allsectionsfont{\mdseries\bfseries}
\setlist[itemize]{noitemsep, topsep=0pt}
%%%%%%%%%%%%%%%%%%%%%%%%%%%%%%%%%%%%%%%%%%%%%%%%%%%%%%%%%%%%%%%%%%%%%%%%
% PDF setup


%%%%%%%%%%%%%%%%%%%%%%%%%%%%%%%%%%%%%%%%%%%%%%%%%%%%%%%%%%%%%%%%%%%%%%%%
% Penalties
%%%%%%%%%%%%%%%%%%%%%%%%%%%%%%%%%%%%%%%%%%%%%%%%%%%%%%%%%%%%%%%%%%%%%%%%

\clubpenalty         = 10000
\displaywidowpenalty = 10000
\widowpenalty        = 10000

%%%%%%%%%%%%%%%%%%%%%%%%%%%%%%%%%%%%%%%%%%%%%%%%%%%%%%%%%%%%%%%%%%%%%%%%
% Main document
%%%%%%%%%%%%%%%%%%%%%%%%%%%%%%%%%%%%%%%%%%%%%%%%%%%%%%%%%%%%%%%%%%%%%%%%

\begin{document}
{
    {
        \Huge \textbf{Luciano Melodia}}\\[0.1cm]
    \emph{Curriculum vitae}\\
    \emph{Last update on \today}.
    
    \begin{flushleft}
        \scriptsize
        \begin{minipage}{0.3\textwidth}
            {\footnotesize \faEnvelope} \hspace{0.1cm} \href{mailto:melodia.luciano@proton.me}{melodia.luciano@proton.me}\\[0.05cm]
            {\footnotesize \faGithub} \hspace{0.15cm} \href{https://github.com/karhunenloeve}{karhunenloeve}\\[0.05cm]
            {\footnotesize \aiarXiv} \hspace{0.1cm} \href{https://arxiv.org/a/melodia_l_1}{melodia\_l\_1}\\[0.05cm]
            {\footnotesize \aiOrcid} \hspace{0.1cm} \href{https://orcid.org/0000-0002-7584-7287}{0000-0002-7584-7287}
        \end{minipage}
        \begin{minipage}{0.4\textwidth}
            {\footnotesize \faPhone} \hspace{0.15cm} +49 175 3372526 \\[0.05cm]
            {\footnotesize \faMapPin} \hspace{0.23cm} Memelstraße 43, 91058 Erlangen\\[0.05cm]
            {\footnotesize \faMars} \hspace{0.1cm} Born on 27.03.1994\\[0.05cm]
            {\footnotesize \faMapO} \hspace{0.1cm} Bavaria, Germany
        \end{minipage}
    \end{flushleft}
}

%%%%%%%%%%%%%%%%%%%%%%%%%%%%%%%%%%%%%%%%%%%%%%%%%%%%%%%%%%%%%%%%%%%%%%%%
\section*{Professions}
%%%%%%%%%%%%%%%%%%%%%%%%%%%%%%%%%%%%%%%%%%%%%%%%%%%%%%%%%%%%%%%%%%%%%%%%
\years{2024--25} Private tutor in German and mathematics.
\begin{itemize}
    \item Abitur preparation for 8 students from the Bavarian Gymnasium in maths.
    \item Secondary school (10th grade) preparation for 1 student in German and maths.
\end{itemize}
\years{2023--24} Tutor Algebra and Geometry, Representation Theory and Operator Algebras, FAU\footnote{Friedrich-Alexander University Erlangen-Nürnberg.}.
\begin{itemize}
    \item Planning and implementation of the exercise lesson in \glqq Topology and Applications\grqq.
    \item Planning and implementation of the exercise lesson in \glqq Topology\grqq.
    \item Planning and implementation of the exercise lesson in \glqq Linear algebra I\grqq.
    \item Planning and conducting intensive proof lessons.
    \item Examination supervision and correction.
\end{itemize}
\years{2021--22} Working student Corscience GmbH \& Co. KG, Erlangen.
\begin{itemize}
    \item Automatic detection of calibration spikes and time stamps in ECG data.
    \item Detection of multiple ECG curves on documents.
    \item Image segmentation using machine learning.
\end{itemize}
\years{2019--21} Researcher Siemens Energy AG, Erlangen.
\begin{itemize}
    \item Programming with CUDA v.11.0, Tensorflow 2.4, CuDNN v.8.0.4.
    \item Programming in Python v.3.8 and v.3.9.
    \item Work with Ubuntu 20.04, Solus 4, Archlinux 5.11.
    \item Implementation and use of convolutional nets, LSTM nets, residual nets, autoencoders, topological autoencoders, and Boltzmann machines for processing time series.
\end{itemize}
\years{2018--21} Researcher Chair of Computer Science 6, FAU.
\begin{itemize}
    \item Correction of written exams and assistance in oral exams.
    \item Self-directed preparation and execution of e-exams.
    \item Corrections to module descriptions for the Data Science program.
\end{itemize}
\years{2015--18} Data scientist at mb Support GmbH, Regensburg.
\begin{itemize}
    \item Implementation of a document pipeline for mass digitization of handwritten documents using neural networks and incorporation into the database application openVIVA.
    \item Installation of the telecommunication interface ASTERISK in openVIVA.
    \item Induction of new employees in openVIVA.
    \item Statistical data and market analysis.
\end{itemize}
\years{2013--15} Research assistant Chair of German Linguistics, Regensburg University.
\begin{itemize}
    \item Examination correction, correction of books and texts.
    \item Website maintenance.
    \item Organization and conduct of conferences.
    \item Implementation of the punc.space web platform.
\end{itemize}
\years{2012--15} Chef in event gastronomy at Apostelkeller, Regensburg.
\begin{itemize}
    \item Cooking according to fixed menu for up to 140 guests.
    \item Waitressing and stock management.
\end{itemize}
\years{2012--15} Staff-based services Trademarketing Service GmbH, Salzgitter. 
\begin{itemize}
    \item Goods management and ordering.
    \item Goods admission.
\end{itemize}
\years{2012--14} Translator at Anatol GmbH \& Co. KG, Regensburg.
\begin{itemize}
    \item Italian -- German translation.
    \item Polish -- German translation.
    \item English -- German translation.
\end{itemize}
\years{2010} Compassion project, Alten- und Pflegeheim St. Josef, Regensburg.

%%%%%%%%%%%%%%%%%%%%%%%%%%%%%%%%%%%%%%%%%%%%%%%%%%%%%%%%%%%%%%%%%%%%%%%%
\section*{Academic Work}
%%%%%%%%%%%%%%%%%%%%%%%%%%%%%%%%%%%%%%%%%%%%%%%%%%%%%%%%%%%%%%%%%%%%%%%%
\years{Teaching}
\vspace{-2pt}
\begin{itemize}[noitemsep, leftmargin=*]
    \item Department of Mathematics, Friedrich-Alexander Universität Erlangen-Nürnberg
          \begin{itemize}
            \item 2024 Exercises in Topology and Applications
            \item 2024 Exercises in Linear Algebra I
            \item 2023 Exercises in Topology
          \end{itemize}
    \item Department of Computer Science, Friedrich-Alexander Universität Erlangen-Nürnberg
          \begin{itemize}
            \item 2021 Lecture Knowledge Disovery in Databases
            \item 2020 Seminar Persistent Homology in Data AnalyticsSeminar 
            \item 2020 Seminar Topological Data Analysis
            \item 2019-21 Exercises Process Oriented Information Systems
            \item 2018-21 Seminar New Technologies in Data Management
            \item 2018-21 Exercises Computer Science for Engineers
            \item 2018 Exercises Conceptional Modeling
          \end{itemize}
\end{itemize}
\vspace{10pt}

\years{Conferences}
\vspace{-2pt}
\begin{itemize}[noitemsep, leftmargin=*]
    \item Learning on Graphs (LOG, 2023, 2022)
    \item $15^{\text{th}}$ International Conference on Advances in Databases, Knowledge,
          and Data Applications (DBKDA, 2023)
    \item International Conference on Learning Representations (ICLR, 2022)
    \item Machine Learning for Irregular Time Series (ML4ITS, 2021)
    \item International Conference on Pattern Recognition (ICPR, 2021)
    \item Topological Data Analysis and Beyond (NeuRIPS, 2020)
    \item International Conference on Practical Mathematical Discourse (ICPMD, 2020)
    \item International Workshop on Combinatorial Image Analysis (IWCIA, 2020)
    \item European Conference on Machine Learning and Principles and Practice of Knowledge Discovery in Databases (ECML/PKDD, 2020, 2019)
    \item Symposium on Principles of Database Systems (SIGMOD/PODS, 2019)
    \item Kolloquium zum Sprachmanagement (2017)
    \item Destandardisierung und Standardvarietät (2013)
\end{itemize}
\vspace{10pt}

\years{Service}
\vspace{-2pt}
\begin{itemize}[noitemsep, leftmargin=*]
    \item Reviewer Learning on Graphs (LOG, 2023, 2022)
    \item Reviewer International Conference on Advances in Databases, Knowledge,
          and Data Applications (DBKDA, 2023, 2020)
    \item Reviewer Geometrical and Topological Representation Learning (ICLR, 2022)
    \item Reviewer Topological Data Analysis and Beyond (NeuRIPS, 2021)
    \item Member Gesellschaft für Informatik e.V. (2019 -- 20)
    \item Member Computational Intelligence and Machine Learning Group (2017 -- 18)
    \item Student Representative Information Science (Regensburg University, 2016)
\end{itemize}
\vspace{10pt}

\years{Supervision} B.Sc. D. Hahn (2021):\\
\emph{Classification of Sensor Signals from Power Plants.} \\
M.Sc. C. Sauerhammer (2021): \\
\emph{A Classification Dashboard for Sensor Signals from Power Plants.} \\
B.Sc. J. Schäfer (2021): \\
\emph{Learning Validation Models from Sensors of a Power Plant.} \\
M.Sc. M. Seidel (2020): \\ 
\emph{Classification of Microbes using Time Series Gas Sensor Array Data.} \\
M.Sc. M.R. Siddiqui (2020): \\
\emph{Extraction of Fetal and Maternal Heartbeats from ECG Signals.}

%%%%%%%%%%%%%%%%%%%%%%%%%%%%%%%%%%%%%%%%%%%%%%%%%%%%%%%%%%%%%%%%%%%%%%%%
\section*{Papers}
%%%%%%%%%%%%%%%%%%%%%%%%%%%%%%%%%%%%%%%%%%%%%%%%%%%%%%%%%%%%%%%%%%%%%%%%
\begin{hangparas}{1em}{1}
    \years{2023 \href{https://karhunenloeve.github.io/TopoHom/main.pdf}{\faFilePdfO}} \textbf{Luciano Melodia}: Notes on Simplicial and Singular Homology. Friedrich-Alexander Universität Erlangen-Nürnberg, Preprint.
\end{hangparas}
%
\begin{hangparas}{1em}{1}
    \years{2021 \href{https://doi.org/10.1007/978-3-030-93736-2_22}{\aiDoi} \href{https://arxiv.org/pdf/2106.02493}{\aiarXiv}} \textbf{Luciano Melodia} and \underline{Richard Lenz}: Homological Time Series Analysis of Sensor Signals from Power Plants. Machine Learning for Irregular Time Series. Machine Learning and Principles and Practice of Knowledge Discovery in Databases. In \underline{Michael Kamp}, \underline{Irena Koprinska}, \underline{Adrien Bibal} et al. (ed.):  Communications in Computer and Information Science. Springer Nature, Switzerland.
\end{hangparas}
%
\begin{hangparas}{1em}{1}
    \years{2021 \href{https://link.springer.com/10.1007/978-3-030-68821-9_2}{\aiDoi} \href{https://arxiv.org/pdf/2004.02881}{\aiarXiv}} \textbf{Luciano Melodia} and \underline{Richard Lenz}: Estimate of the Neural Network Dimension Using Algebraic Topology and Lie Theory. Image Mining. Theory and Applications VII. Pattern Recognition and Information Forensics. In \underline{Alberto Del Bimbo}, \underline{Rita Cucchiara}, \underline{Stan Sciaroff} et al. (ed.): Lecture Notes in Computer Science. Springer Nature, Switzerland.
\end{hangparas}
%
\begin{hangparas}{1em}{1}
    \years{2020 \href{https://link.springer.com/chapter/10.1007\%2F978-3-030-51002-2_3}{\aiDoi} \href{https://arxiv.org/pdf/1911.02922}{\aiarXiv}} \textbf{Luciano Melodia} and \underline{Richard Lenz}: Persistent Homology as Stopping-Criterion for Voronoi Interpolation. Proceedings of the International Workshop on Combinatorial Image Analysis. In \underline{Tibor Lukić}, \underline{Reneta Barneva}, \underline{Valentin Brimkov} et al. (ed.): Lecture Notes in Computer Science. Springer, Cham.
\end{hangparas}
%
\begin{hangparas}{1em}{1}
    \years{2018 \href{https://osf.io/zp6nv/}{\aiDoi} \href{https://arxiv.org/pdf/1805.09108}{\aiarXiv}} \textbf{Luciano Melodia}: Deep Learning Schätzung zur absorbierten Strahlungsdosis für die nuklearmedizinische Diagnostik. Library of the \underline{University of Regensburg}, \underline{Master Thesis} in Information Science.
\end{hangparas}
%
\begin{hangparas}{1em}{1}
    \years{2015 \href{https://www.logos-verlag.de/cgi-bin/engbuchmid?isbn=3808&lng=deu}{\aiDoi} \href{https://ling.auf.net/lingbuzz/004798}{\faFilePdfO}} \textbf{Luciano Melodia}: Zur Verwendung des Paradigmas brauchen mit und ohne zu mit Infinitiv. In \underline{Kate\v{s}ina \v{S}ichov\`{a}}, \underline{Reinhard Krapp}, \underline{Rössler Paul} et al. (ed.): Standardvarietät des Deutschen -- Fallbeispiele aus der sozialen Praxis, Logos, Berlin.
\end{hangparas}
%
\vspace{0.5cm}

%%%%%%%%%%%%%%%%%%%%%%%%%%%%%%%%%%%%%%%%%%%%%%%%%%%%%%%%%%%%%%%%%%%%%%%%
\section*{Education}
%%%%%%%%%%%%%%%%%%%%%%%%%%%%%%%%%%%%%%%%%%%%%%%%%%%%%%%%%%%%%%%%%%%%%%%%
\years{2024 -- 26} \emph{Mathematics} \textbf{M.Sc.}\\
Minor: italian philology, Friedrich-Alexander University Erlangen-Nürnberg. \\
\years{2021 -- 24} \emph{Mathematics} \textbf{B.Sc.}\\
Thesis: "Algebraic and topological persistence". \\
Minor: computer science, Friedrich-Alexander University Erlangen-Nürnberg. \\
\years{2015 -- 18} \emph{Information Science} \textbf{M.A.} \\
Thesis: "Deep learning estimation of absorbed radiation dose for nuclear medicine diagnostics".\\
Minor: biophysics, Regensburg University. \\
\years{2012 -- 15} \emph{German Philology} \textbf{B.A.}\\
Thesis: "Development of an application for punctuation research using linguistic modules for information retrieval".\\
Majors: italian philology, information science, media informatics, Regensburg University. \\
\years{2012 -- 13} \emph{Web Developer}, Rechenzentrum Regensburg University. \\
\years{2012} \emph{Abitur}, Albertus-Magnus-Gymnasium, Regensburg.

%%%%%%%%%%%%%%%%%%%%%%%%%%%%%%%%%%%%%%%%%%%%%%%%%%%%%%%%%%%%%%%%%%%%%%%%
\section*{Interests}
%%%%%%%%%%%%%%%%%%%%%%%%%%%%%%%%%%%%%%%%%%%%%%%%%%%%%%%%%%%%%%%%%%%%%%%%
\years{Coding} Python, JavaScript.\\
\years{Software} GUDHI, Dionysus, Keras.\\
\years{Languages} German (native), English C2, Italian C2, Polish B2.\\
\years{Hobbies} Cooking, Reading.\\
\years{Sports} Muay Thai.
\end{document}