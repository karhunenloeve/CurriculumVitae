\documentclass[a4paper,11pt]{article}

\usepackage[british]{babel}
\usepackage{fontspec}
\usepackage{fontawesome}
\usepackage{geometry}
\usepackage{marginnote}
\usepackage{scrextend}
\usepackage{sectsty}
\usepackage{microtype}
\usepackage[usenames,dvipsnames]{color}
\usepackage{academicons}
\usepackage{hanging}
\usepackage{enumitem}
\usepackage{tikz}
\usepackage{classicthesis}
\usepackage[export]{adjustbox}
\geometry{
  a4paper,
  textwidth      = 14.0cm,
  textheight     = 25.0cm,
  marginparwidth =  2.5cm
}
\defaultfontfeatures{Ligatures=TeX}
\setlength{\parindent}{0pt}
\setlength{\marginparsep}{10pt}

\setlength{\parindent}{0pt}
\setlength{\skip\footins}{0.5cm}

\definecolor{amber}{rgb}{1.0, 0.75, 0.0}
\definecolor{bittersweet}{rgb}{1.0, 0.44, 0.37}
\definecolor{darkcyan}{rgb}{0.0, 0.55, 0.55}
\definecolor{darkmagenta}{rgb}{0.55, 0.0, 0.55}
\definecolor{denim}{rgb}{0.08, 0.38, 0.74}

\NewDocumentCommand{\statcirc}{ O{#2} m }{%
    \begin{tikzpicture}[every node/.style={inner sep=0,outer sep=-1}]
    \fill[#2] (0,0) circle (0.55ex);
    \fill[#1] (0,0) -- (180:0.55ex) arc (180:0:0.55ex) -- cycle;
    \end{tikzpicture}
}
\newcommand  {\rd}{\textsuperscript{\textup{rd}}\xspace}
\newcommand  {\nd}{\textsuperscript{\textup{nd}}\xspace}
\renewcommand{\th}{\textsuperscript{\textup{th}}\xspace}
\newcommand{\authorequal}{\kern-0.1em\textsuperscript{\dagger}}
\renewcommand*{\raggedleftmarginnote}{}
\reversemarginpar
\newcommand{\years}[1]{\marginnote{\scriptsize #1}}

\allsectionsfont{\mdseries\bfseries}
\setlist[itemize]{noitemsep, topsep=0pt}
%%%%%%%%%%%%%%%%%%%%%%%%%%%%%%%%%%%%%%%%%%%%%%%%%%%%%%%%%%%%%%%%%%%%%%%%
% PDF setup


%%%%%%%%%%%%%%%%%%%%%%%%%%%%%%%%%%%%%%%%%%%%%%%%%%%%%%%%%%%%%%%%%%%%%%%%
% Penalties
%%%%%%%%%%%%%%%%%%%%%%%%%%%%%%%%%%%%%%%%%%%%%%%%%%%%%%%%%%%%%%%%%%%%%%%%

\clubpenalty         = 10000
\displaywidowpenalty = 10000
\widowpenalty        = 10000

%%%%%%%%%%%%%%%%%%%%%%%%%%%%%%%%%%%%%%%%%%%%%%%%%%%%%%%%%%%%%%%%%%%%%%%%
% Main document
%%%%%%%%%%%%%%%%%%%%%%%%%%%%%%%%%%%%%%%%%%%%%%%%%%%%%%%%%%%%%%%%%%%%%%%%

\begin{document}
{
  {
  \Huge \textbf{Luciano Melodia}}\\[0.1cm]
  \emph{Curriculum vitae}\\
  \emph{Last update on \today}.

  \begin{flushleft}
    \scriptsize
      \begin{minipage}{0.3\textwidth}
        {\footnotesize \faEnvelope} \hspace{0.1cm} \href{mailto:l.melodia@pm.me}{l.melodia@pm.me}\\[0.05cm]
        {\footnotesize \faGithub} \hspace{0.15cm} \href{https://github.com/karhunenloeve}{karhunenloeve}\\[0.05cm]
        {\footnotesize \aiarXiv} \hspace{0.1cm} \href{https://arxiv.org/a/melodia_l_1}{melodia\_l\_1}\\[0.05cm]
        {\footnotesize \aiOrcid} \hspace{0.1cm} \href{https://orcid.org/0000-0002-7584-7287}{0000-0002-7584-7287}
      \end{minipage}
      \begin{minipage}{0.4\textwidth}
        {\footnotesize \faPhone} \hspace{0.15cm} +49 175 3372526 \\[0.05cm]
        {\footnotesize \faMapPin} \hspace{0.23cm} Gebbertstraße 95, 91058 Erlangen\\[0.05cm]
        {\footnotesize \faMars} \hspace{0.1cm} Born on 27.03.1994\\[0.05cm]
        {\footnotesize \faMapO} \hspace{0.1cm} Bavaria, Germany
      \end{minipage}
  \end{flushleft}
}

%%%%%%%%%%%%%%%%%%%%%%%%%%%%%%%%%%%%%%%%%%%%%%%%%%%%%%%%%%%%%%%%%%%%%%%%
\section*{Professions}
%%%%%%%%%%%%%%%%%%%%%%%%%%%%%%%%%%%%%%%%%%%%%%%%%%%%%%%%%%%%%%%%%%%%%%%%
\years{2023} \href{}{Mathematics and computer science tutoring}, Erlangen.\\
\years{2021--22} \href{}{Working student} Corscience GmbH \& Co. KG, Erlangen.\\
\years{2020--21} Researcher Professorship for Evolutionary Data Management, FAU Erlangen-Nürnberg.\\
\years{2018--21} Researcher Chair of Computer Science 6, FAU Erlangen-Nürnberg.\\
\years{2020--21} Researcher Siemens Energy AG, Erlangen.\\
\years{2019--20} Researcher Siemens Gas and Power GmbH \& Co. KG, Erlangen.\\
\years{2017--18} Data scientist at mb Support GmbH, Regensburg.\\
\years{2015--17} Working student mb Support GmbH, Regensburg.\\
\years{2014--15} Research assistant Chair of German Linguistics, Regensburg University.\\
\years{2013--14} Student assistant Chair of German Linguistics, Regensburg University.\\
\years{2012--15} Chef in event gastronomy Apostelkeller, Regensburg.\\
\years{2012--15} Staff-based services Trademarketing Service GmbH, Salzgitter. \\
\years{2012--14} Translator for Italian and English at Anatol GmbH \& Co. KG, Regensburg.\\
\years{2010} Compassion project, Alten- und Pflegeheim St. Josef, Regensburg.

%%%%%%%%%%%%%%%%%%%%%%%%%%%%%%%%%%%%%%%%%%%%%%%%%%%%%%%%%%%%%%%%%%%%%%%%
\section*{Academic Work}
%%%%%%%%%%%%%%%%%%%%%%%%%%%%%%%%%%%%%%%%%%%%%%%%%%%%%%%%%%%%%%%%%%%%%%%%
\years{Lectures} Knowledge Disovery in Databases (FAU, 2021) \bullet{} Persistent Homology in Data Analytics Seminar (FAU, 2020) \bullet{} Topological Data Analysis Seminar (FAU, 2020) \bullet{} Process Oriented Information Systems (FAU, 2019 -- 20) \bullet{} New Technologies in Data Management (FAU, 2018 -- 21) \bullet{} Computer Science for Engineers (FAU, 2018 -- 20) \bullet{} Big Data Seminar (FAU, 2018 -- 19) \bullet{} Conceptional Modeling (FAU, 2018). \\
\years{Conferences} Learning on Graphs (LOG, 2022) \bullet{} International Conference on Learning Representations (ICLR, 2022) \bullet{} Machine Learning for Irregular Time Series (ML4ITS, 2021), Bilbao Spain \bullet{} International Conference on Pattern Recognition (ICPR, 2021), Milano Italy \bullet{} Topological Data Analysis and Beyong (NeuRIPS, 2020), Vancouver, Canada \bullet{} International Conference on Practical Mathematical Discourse (ICPMD, 2020) Kerala, India \bullet{} International Workshop on Combinatorial Image Analysis (IWCIA, 2020) Novi Sad, Serbia \bullet{} European Conference on Machine Learning and Principles and Practice of Knowledge Discovery in Databases (ECML/PKDD, 2019) Würzburg, Germany  (ECML/PKDD, 2020) Gent, Belgium \bullet{} Symposium on Principles of Database Systems (SIGMOD/PODS, 2019) Amsterdam, Netherlands \bullet{} Kolloquium zum Sprachmanagement (2017), Regensburg, Germany \bullet{} Destandardisierung und Standardvarietät (2013) Prague, Czech Republic. \\
\years{Service} Reviewer Learning on Graphs (LOG, 2022) \bullet{} Reviewer Geometrical and Topological Representation Learning (ICLR, 2022) \bullet{} Reviewer TDA and Beyond (NeuRIPS, 2021) \bullet{} Reviewer (ECML/PKDD, 2020) \bullet{} Member Gesellschaft für Informatik e.V. (2019 -- 20) \bullet{} Member Computational Intelligence and Machine Learning Group (2017 -- 18) \bullet{} Student Representative Information Science (Regensburg University, 2016).\\
\years{Supervision} B.Sc. D. Hahn (2021):\\
\emph{Classification of Sensor Signals from Power Plants.} \\
M.Sc. C. Sauerhammer (2021): \\
\emph{A Classification Dashboard for Sensor Signals from Power Plants.} \\
B.Sc. J. Schäfer (2021): \\
\emph{Learning Validation Models from Sensors of a Power Plant.} \\
M.Sc. M. Seidel (2020): \\ 
\emph{Classification of Microbes using Time Series Gas Sensor Array Data.} \\
M.Sc. M.R. Siddiqui (2020): \\
\emph{Extraction of Fetal and Maternal Heartbeats from ECG Signals.}

%%%%%%%%%%%%%%%%%%%%%%%%%%%%%%%%%%%%%%%%%%%%%%%%%%%%%%%%%%%%%%%%%%%%%%%%
\section*{Papers}
%%%%%%%%%%%%%%%%%%%%%%%%%%%%%%%%%%%%%%%%%%%%%%%%%%%%%%%%%%%%%%%%%%%%%%%%
\begin{hangparas}{1em}{1}
\years{2021, \href{https://doi.org/10.1007/978-3-030-93736-2_22}{\aiDoi} \href{https://arxiv.org/pdf/2106.02493}{\aiarXiv}} \textbf{Luciano Melodia} and \underline{Richard Lenz}: Homological Time Series Analysis of Sensor Signals from Power Plants. Machine Learning for Irregular Time Series. Machine Learning and Principles and Practice of Knowledge Discovery in Databases. In \underline{Michael Kamp}, \underline{Irena Koprinska}, \underline{Adrien Bibal} et al. (ed.):  Communications in Computer and Information Science. Springer Nature, Switzerland.
\end{hangparas}
%
\begin{hangparas}{1em}{1}
\years{2021, \href{https://link.springer.com/10.1007/978-3-030-68821-9_2}{\aiDoi} \href{https://arxiv.org/pdf/2004.02881}{\aiarXiv}} \textbf{Luciano Melodia} and \underline{Richard Lenz}: Estimate of the Neural Network Dimension Using Algebraic Topology and Lie Theory. Image Mining. Theory and Applications VII. Pattern Recognition and Information Forensics. In \underline{Alberto Del Bimbo}, \underline{Rita Cucchiara}, \underline{Stan Sciaroff} et al. (ed.): Lecture Notes in Computer Science. Springer Nature, Switzerland.
\end{hangparas}
%
\begin{hangparas}{1em}{1}
\years{2020, \href{https://link.springer.com/chapter/10.1007\%2F978-3-030-51002-2_3}{\aiDoi} \href{https://arxiv.org/pdf/1911.02922}{\aiarXiv}} \textbf{Luciano Melodia} and \underline{Richard Lenz}: Persistent Homology as Stopping-Criterion for Voronoi Interpolation. Proceedings of the International Workshop on Combinatorial Image Analysis. In \underline{Tibor Lukić}, \underline{Reneta Barneva}, \underline{Valentin Brimkov} et al. (ed.): Lecture Notes in Computer Science. Springer, Cham.
\end{hangparas}
%
\begin{hangparas}{1em}{1}
\years{2018, \href{https://osf.io/zp6nv/}{\aiDoi} \href{https://arxiv.org/pdf/1805.09108}{\aiarXiv}} \textbf{Luciano Melodia}: Deep Learning Schätzung zur absorbierten Strahlungsdosis für die nuklearmedizinische Diagnostik. Library of the \underline{University of Regensburg}, \underline{Master Thesis} in Information Science.
\end{hangparas}
%
\begin{hangparas}{1em}{1}
\years{2015, \href{https://www.logos-verlag.de/cgi-bin/engbuchmid?isbn=3808&lng=deu}{\aiDoi} \href{https://ling.auf.net/lingbuzz/004798}{\aiarXiv}} \textbf{Luciano Melodia}: Zur Verwendung des Paradigmas brauchen mit und ohne zu mit Infinitiv. In \underline{Kate\v{s}ina \v{S}ichov\`{a}}, \underline{Reinhard Krapp}, \underline{Rössler Paul} et al. (ed.): Standardvarietät des Deutschen -- Fallbeispiele aus der sozialen Praxis, Logos, Berlin.
\end{hangparas}
%
\vspace{0.5cm}

%%%%%%%%%%%%%%%%%%%%%%%%%%%%%%%%%%%%%%%%%%%%%%%%%%%%%%%%%%%%%%%%%%%%%%%%
\section*{Education}
%%%%%%%%%%%%%%%%%%%%%%%%%%%%%%%%%%%%%%%%%%%%%%%%%%%%%%%%%%%%%%%%%%%%%%%%
\years{2015 -- 18} Master of Arts (M.A.) Information Science, Regensburg University. \\
\years{2021 -- 24} Bachelor of Science (B.Sc.) Mathematics, Computer Science, Friedrich-Alexander University Erlangen-Nürnberg. \\
\years{2012 -- 15} Bachelor of Arts (B.A.) German Philology, Italian Philology, Information Science and Media Informatics, Regensburg University. \\
\years{2012 -- 13} Web Developer, Rechenzentrum Regensburg University. \\
\years{2012} Abitur, Albertus-Magnus-Gymnasium, Regensburg. \\
\years{Certificates} Data Visualization with ggplot II \bullet{} Data Visualization with ggplot I \bullet{} Deep learning in Python \bullet{} Text Mining: Bag of Words \bullet{} Introduction to Machine Learning \bullet{} Machine Learning Toolbox \bullet{} Intro to Python for Data Science \bullet{} Exploratory Data Analysis \bullet{} Introduction to R \bullet{} Supervised learning in R: Regression \bullet{} Supervised Learning in R: Classification \bullet{} Credit Risk Modeling in R \bullet{} Data visualization in R \bullet{} Intermediate R: Practice Course \bullet{} Intermediate R \bullet{} Discrete Mathematics \bullet{} Mathematics for Machine Learning: Linear Algebra \bullet{} Mathematics for Machine Learning: Multivariate Calculus \bullet{} Machine Learning from a Mathematical Viewpoint \bullet The Python Mega Course: Build 10 Real World Applications \bullet{} The Rust Programming Language \bullet{} Evolutionary Game Theory \bullet{}
Introduction to Complex Analysis.

%%%%%%%%%%%%%%%%%%%%%%%%%%%%%%%%%%%%%%%%%%%%%%%%%%%%%%%%%%%%%%%%%%%%%%%%
\section*{Interests}
%%%%%%%%%%%%%%%%%%%%%%%%%%%%%%%%%%%%%%%%%%%%%%%%%%%%%%%%%%%%%%%%%%%%%%%%
\years{Code} Python, Rust, \LaTeX, CSS, HTML, XML, JavaScript, PHP, C\#, Java.\\
\years{Software} GUDHI, Dionysus, Keras, Theano and Tensorflow.\\
\years{Language} German (native), English C2, Italian C2, Polish B2 and Spanish A2.\footnote{Language levels are self estimates according to the \href{https://en.wikipedia.org/wiki/Common_European_Framework_of_Reference_for_Languages}{CEFR} standard.}\\
\years{Math} Homology \& Cohomology, Commutative Algebra.\\
\years{Computer science} Topological Data Analysis, Neural Networks.\\
\years{Sports}
{\Large\circ} {\Large\color{amber}\circ} {\Large\color{bittersweet}\circ} {\Large\color{darkcyan}\circ} {\Large\color{denim}\circ} {\Large\color{darkmagenta}\circ} Pra Jiad in Muay Thai (2020 -- 23).\\
{\Large\color{amber}\bullet} {\Large\color{bittersweet}\bullet} {\Large\color{darkcyan}\bullet} Sash in Weng Chun (2020 -- 23).\\
%Win against Daniel Lee, Lamai Beach $\leq 76$ kg, Koh Samui, Thailand (2023).\\
Member WMC Lamai Muay Thai Camp, Koh Samui, Thailand (2023).\\
Weng Chun Teaching (substitution), Kung Fu Zentrum Erlangen, Germany (2023).\\
Weng Chun Summer Camp, Bamberg, Germany (2021).\\
Member Kung Fu Zentrum Erlangen, Germany (2020 -- 21).\\
Member Deutscher Alpenverein e.V. (2011 -- 13).\\
{\Large\circ} {\Large\color{amber}\circ} {\Large\color{amber}\bullet} {\Large\color{bittersweet}\circ} {\Large\color{bittersweet}\bullet} {\Large\color{darkcyan}\circ} {\Large\color{darkcyan}\bullet} Belt in Judo (2006 -- 10).\\
Member SG Walhalla Regensburg e.V. (2006 -- 10).
\end{document}
