\documentclass[a4paper, 11pt]{article}
\usepackage[british]{babel}
\usepackage{classicthesis}
\usepackage{fontspec}

\setmainfont[Ligatures={Required,Common,Contextual,Rare,Historic,TeX},Numbers=OldStyle,RawFeature={+ss05,+dlig,+hlig,+calt,+liga},ItalicFeatures={RawFeature={+cv04,+clig,+swsh,+calt,+liga,+hlig,+ss05},CharacterVariant=5:0}]{EB Garamond}
\usepackage{amsmath}
\usepackage{fontawesome}
\usepackage{geometry}
\usepackage{marginnote}
\usepackage{sectsty}
\usepackage{academicons}
\usepackage{hanging}
\usepackage{enumitem}
\usepackage{etaremune}
\usepackage{hyperref}
\geometry{a4paper, textwidth=11cm, textheight=25.0cm, marginparwidth=2.5cm}
\setlength{\parindent}{0pt}
\setlength{\marginparsep}{10pt}
\setlength{\skip\footins}{0.5cm}

\setlist[enumerate]{nosep}
\newcommand{\rd}{\textsuperscript{\textup{rd}}\xspace}
\newcommand{\nd}{\textsuperscript{\textup{nd}}\xspace}
\renewcommand{\th}{\textsuperscript{\textup{th}}\xspace}
\newcommand{\authorequal}{\kern-0.1em\textsuperscript{\dagger}}
\renewcommand*{\raggedleftmarginnote}{}
\reversemarginpar
\newcommand{\years}[1]{\marginnote{\scriptsize #1}}

\allsectionsfont{\mdseries\bfseries}
\setlist[itemize]{noitemsep, topsep=0pt}

%%%%%%%%%%%%%%%%%%%%%%%%%%%%%%%%%%%%%%%%%%%%%%%%%%%%%%%%%%%%%%%%%%%%%%%%
% Main document
%%%%%%%%%%%%%%%%%%%%%%%%%%%%%%%%%%%%%%%%%%%%%%%%%%%%%%%%%%%%%%%%%%%%%%%%

\begin{document}
{\Huge \textbf{Luciano Melodia}}\\[0.1cm]
\emph{Curriculum vitae}\\ 
\emph{Last update on \today}.

\begin{flushleft}
	\scriptsize
	\begin{minipage}{0.3\textwidth}
		{\footnotesize \faEnvelope} \hspace{0.1cm} \href{mailto:luciano.melodia@fau.de}{luciano.melodia@fau.de}\\[0.05cm]
		{\footnotesize \faGithub} \hspace{0.15cm} \href{https://github.com/karhunenloeve}{karhunenloeve}\\[0.05cm]
		{\footnotesize \aiarXiv} \hspace{0.1cm} \href{https://arxiv.org/a/melodia_l_1}{melodia\_l\_1}\\[0.05cm]
		{\footnotesize \aiOrcid} \hspace{0.1cm} \href{https://orcid.org/0000-0002-7584-7287}{0000-0002-7584-7287}
	\end{minipage}
	\begin{minipage}{0.4\textwidth}
		{\footnotesize \faPhone} \hspace{0.20cm} +49 175 3372526 \\[0.05cm]
		{\footnotesize \faMapO} \hspace{0.1cm} Regensburg, Germany
	\end{minipage}
\end{flushleft}

%%%%%%%%%%%%%%%%%%%%%%%%%%%%%%%%%%%%%%%%%%%%%%%%%%%%%%%%%%%%%%%%%%%%%%%%
\section*{Professions}
%%%%%%%%%%%%%%%%%%%%%%%%%%%%%%%%%%%%%%%%%%%%%%%%%%%%%%%%%%%%%%%%%%%%%%%%
\years{2023--25} Student Assistant at Algebra and Geometry, \\
Representation Theory and Operator Algebras and \\
Applied Analysis FAU\footnote{Friedrich-Alexander University Erlangen-Nürnberg.}.
\begin{itemize}
	\item Tutor in \glqq Analysis III\grqq.
	\item Tutor in \glqq Topology and Applications\grqq.
	\item Tutor in \glqq Topology\grqq.
	\item Lecture representation of Prof. Li (Tietze Extension Theorem).
	\item Tutor in \glqq Linear Algebra I\grqq.
	\item Conducting proof lessons.
	\item Conducting exercise lessons.
	\item Examination, supervision and correction.
\end{itemize}
\years{2025} Lecturer in Mathematics at Paukkammer Erlangen.\\
\years{2024--25} Private tutor in German and Mathematics.
\begin{itemize}
	\item Abitur, 12 students, Bavarian Gymnasium.
	\item Secondary school, 2 students, Bavarian Mittelschule.
	\item Elementary school, 1 student.
\end{itemize}
\years{2021--22} Working student at Corscience GmbH \& Co. KG, Erlangen.
\begin{itemize}
	\item Automatic detection of calibration spikes in ECG data.
	\item Detection of multiple ECG curves on documents.
	\item Image segmentation using machine learning.
\end{itemize}
\years{2019--21} Researcher at Siemens Energy AG, Erlangen.
\begin{itemize}
	\item Programming with CUDA v.11.0, Tensorflow 2.4, CuDNN v.8.0.4.
	\item Programming in Python v.3.8 and v.3.9.
	\item Work with Ubuntu 20.04, Solus 4, Archlinux 5.11.
	\item Implementation and use of convolutional nets, LSTM nets, residual nets, autoencoders, topological autoencoders, and Boltzmann machines for processing time series.
\end{itemize}
\years{2018--21} Researcher Chair of Computer Science 6, FAU.
\begin{itemize}
	\item Correction of written exams and assistance in oral exams.
	\item Self-directed preparation and execution of e-exams.
	\item Corrections to module descriptions for the Data Science program.
	\item Planning and implementation of the
	      \begin{itemize}
	      	\item lecture \glqq Knowledge Discovery in Databases\grqq.
	      	\item seminar \glqq Persistent Homology in Data Analytics\grqq.
	      	\item seminar \glqq Topological Data Analysis\grqq.
	      	\item seminar \glqq New Technologies in Data Management\grqq.
	      	\item exercise lessons in \glqq Process Oriented Information Systems\grqq.
	      	\item exercise lessons in \glqq Computer Science for Engineers\grqq.
	      	\item exercise lessons in \glqq Conceptual Modeling\grqq.
	      \end{itemize}
\end{itemize}
\years{2015--18} Data scientist at mb Support GmbH, Regensburg.
\begin{itemize}
	\item Implementation of a document pipeline for mass digitization of handwritten documents using neural networks and incorporation into the database application openVIVA.
	\item Integration of the telecommunication interface ASTERISK.
	\item Induction of new employees into openVIVA.
	\item Statistical data and market analysis.
\end{itemize}
\years{2013--15} Research assistant Chair of German Linguistics, Regensburg University.
\begin{itemize}
	\item Examination correction, correction of books and texts.
	\item Website maintenance.
	\item Organization and conduct of conferences.
	\item Implementation of the punc.space web platform.
\end{itemize}
\years{2012--15} Chef in event gastronomy at Apostelkeller, Regensburg.
\begin{itemize}
	\item Cooking according to a fixed menu for up to 140 guests.
	\item Waitressing and stock management.
\end{itemize}
\years{2012--15} Staff-based services at Trademarketing Service GmbH, Salzgitter.
\begin{itemize}
	\item Goods management and ordering.
	\item Goods receipt.
\end{itemize}
\years{2012--14} Translator at Anatol GmbH \& Co. KG, Regensburg.
\begin{itemize}
	\item Italian -- German translation.
	\item Polish -- German translation.
	\item English -- German translation.
\end{itemize}
\years{2010} Volunteer at Alten- und Pflegeheim St. Josef, Regensburg.

\newpage
%%%%%%%%%%%%%%%%%%%%%%%%%%%%%%%%%%%%%%%%%%%%%%%%%%%%%%%%%%%%%%%%%%%%%%%%
\section*{Academic Work}
%%%%%%%%%%%%%%%%%%%%%%%%%%%%%%%%%%%%%%%%%%%%%%%%%%%%%%%%%%%%%%%%%%%%%%%%
\years{Teaching}
\vspace{-2pt}
\begin{itemize}[noitemsep, leftmargin=*]
	\item Department of Mathematics, \\
	      Friedrich-Alexander Universität Erlangen-Nürnberg
	      \begin{itemize}
	      	\item 2025 Exercises in Analysis III
	      	\item 2024 Exercises in Topology and Applications
	      	\item 2024 Exercises in Linear Algebra I
	      	\item 2023 Exercises in Topology
	      \end{itemize}
	\item Department of Computer Science, \\
	      Friedrich-Alexander Universität Erlangen-Nürnberg
	      \begin{itemize}
	      	\item 2021 Lecture on Knowledge Discovery in Databases
	      	\item 2021 Exercises in Process Oriented Information Systems
	      	\item 2021 Seminar on New Technologies in Data Management
	      	\item 2021 Exercises in Computer Science for Engineers
	      	\item 2020 Seminar on Persistent Homology in Data Analytics
	      	\item 2020 Seminar on Topological Data Analysis
	      	\item 2020 Exercises in Process Oriented Information Systems
	      	\item 2020 Exercises in Computer Science for Engineers
	      	\item 2020 Seminar on New Technologies in Data Management
	      	\item 2019 Exercises in Computer Science for Engineers
	      	\item 2019 Exercises in Process Oriented Information Systems
	      	\item 2019 Seminar on New Technologies in Data Management
	      	\item 2018 Exercises in Computer Science for Engineers
	      	\item 2018 Seminar on New Technologies in Data Management
	      	\item 2018 Exercises in Conceptual Modeling
	      \end{itemize}
\end{itemize}
\vspace{10pt}

\years{Conferences}
\vspace{-10pt}
\begin{etaremune}[itemsep=-5pt, leftmargin=15pt]
	\item[2024] Learning on Graphs
	\item[2023] Learning on Graphs
	\item[2023] $15^{\text{th}}$ International Conference on Advances in Databases, Knowledge, and Data Applications
	\item[2022] Learning on Graphs
	\item[2022] International Conference on Learning Representations 
	\item[2021] Machine Learning for Irregular Time Series
	\item[2021] International Conference on Pattern Recognition
	\item[2020] Topological Data Analysis and Beyond
	\item[2020] International Conference on Practical Mathematical Discourse
	\item[2020] International Workshop on Combinatorial Image Analysis
	\item[2020] European Conference on Machine Learning and Principles and Practice of Knowledge Discovery in Databases
	\item[2019] European Conference on Machine Learning and Principles and Practice of Knowledge Discovery in Databases
	\item[2019] Symposium on Principles of Database Systems
	\item[2017] Kolloquium zum Sprachmanagement
	\item[2013] Destandardisierung und Standardvarietät
\end{etaremune}
\vspace{10pt}

\years{Service}
\vspace{-10pt}
\begin{etaremune}[itemsep=-5pt, leftmargin=15pt]
	\item[2024] Oskar-Karl-Forster scholarship fellow
	\item[2024] Reviewer for Learning on Graphs
	\item[2024] Student Representative for the Department of Mathematics at the Friedrich-Alexander University Erlangen-Nürnberg
	\item[2024] Reviewer for the International Conference on Advances in Databases, Knowledge, and Data Applications
	\item[2023] Reviewer for Learning on Graphs
	\item[2023] Reviewer for the International Conference on Advances in Databases, Knowledge, and Data Applications
	\item[2022] Reviewer for Learning on Graphs
	\item[2022] Reviewer for the Workshop Geometrical and Topological Representation Learning, International Conference on Learning Representations
	\item[2021] Reviewer for the Workshop Topological Data Analysis and Beyond, Neural Information Processing Systems
	\item[2020] Reviewer for the International Conference on Advances in Databases, Knowledge, and Data Applications
	\item[2020] Member of the Gesellschaft für Informatik e.V.
	\item[2019] Member of the Gesellschaft für Informatik e.V.
	\item[2018] Member of the Computational Intelligence and Machine Learning Group, CIML University Regensburg
	\item[2017] Member of the Computational Intelligence and Machine Learning Group, CIML University Regensburg
	\item[2016] Student Representative for the Department of Language, Literature and Cultural Sciences at the Regensburg University
\end{etaremune}
\vspace{10pt}

\years{Supervision}
\vspace{-10pt}
\begin{etaremune}[itemsep=-5pt, leftmargin=15pt]
	\item B.Sc. Hahn (2021): Classification of Sensor Signals from Power Plants.
	\item M.Sc. Sauerhammer (2021): A Classification Dashboard for Sensor Signals from Power Plants.
	\item B.Sc. Schäfer (2021): Learning Validation Models from Sensors of a Power Plant.
	\item M.Sc. Seidel (2020): Classification of Microbes using Time Series Gas Sensor Array Data.
	\item M.Sc. Siddiqui (2020): Extraction of Fetal and Maternal Heartbeats from ECG Signals.
\end{etaremune}

\newpage
%%%%%%%%%%%%%%%%%%%%%%%%%%%%%%%%%%%%%%%%%%%%%%%%%%%%%%%%%%%%%%%%%%%%%%%%
\section*{Papers}
%%%%%%%%%%%%%%%%%%%%%%%%%%%%%%%%%%%%%%%%%%%%%%%%%%%%%%%%%%%%%%%%%%%%%%%%
\begin{hangparas}
	{1em}{1} \years{2024 \href{https://karhunenloeve.github.io/BscMath/main.pdf}{\faFilePdfO}} \textbf{Luciano Melodia}: Algebraic and Topological Persistence. Bachelor thesis. Friedrich-Alexander Universität Erlangen-Nürnberg.
\end{hangparas}
%
\begin{hangparas}
	{1em}{1} \years{2023 \href{https://karhunenloeve.github.io/TopoHom/main.pdf}{\faFilePdfO}} \textbf{Luciano Melodia}: Notes on Simplicial and Singular Homology. Seminar paper. Friedrich-Alexander Universität Erlangen-Nürnberg.
\end{hangparas}
%
\begin{hangparas}
	{1em}{1} \years{2022 \href{https://karhunenloeve.github.io/TopoSheaf/main.pdf}{\faFilePdfO}} \textbf{Luciano Melodia}: Natürliche Transformationen, Äquivalenzen von Kategorien, darstellbare Funktoren und das Lemma von Yoneda. Seminar paper. Friedrich-Alexander Universität Erlangen-Nürnberg.
\end{hangparas}
%
\begin{hangparas}
	{1em}{1} \years{2021 \href{https://doi.org/10.1007/978-3-030-93736-2_22}{\aiDoi} \href{https://arxiv.org/pdf/2106.02493}{\aiarXiv}} \textbf{Luciano Melodia} and \underline{Richard Lenz}: Homological Time Series Analysis of Sensor Signals from Power Plants. Machine Learning for Irregular Time Series. Machine Learning and Principles and Practice of Knowledge Discovery in Databases. In \underline{Michael Kamp}, \underline{Irena Koprinska}, \underline{Adrien Bibal} et al. (ed.): Communications in Computer and Information Science. Springer Nature, Switzerland.
\end{hangparas}
%
\begin{hangparas}
	{1em}{1} \years{2021 \href{https://link.springer.com/10.1007/978-3-030-68821-9_2}{\aiDoi} \href{https://arxiv.org/pdf/2004.02881}{\aiarXiv}} \textbf{Luciano Melodia} and \underline{Richard Lenz}: Estimate of the Neural Network Dimension Using Algebraic Topology and Lie Theory. Image Mining. Theory and Applications VII. Pattern Recognition and Information Forensics. In \underline{Alberto Del Bimbo}, \underline{Rita Cucchiara}, \underline{Stan Sciaroff} et al. (ed.): Lecture Notes in Computer Science. Springer Nature, Switzerland.
\end{hangparas}
%
\begin{hangparas}
	{1em}{1} \years{2020 \href{https://link.springer.com/chapter/10.1007\%2F978-3-030-51002-2_3}{\aiDoi} \href{https://arxiv.org/pdf/1911.02922}{\aiarXiv}} \textbf{Luciano Melodia} and \underline{Richard Lenz}: Persistent Homology as a Stopping Criterion for Voronoi Interpolation. Proceedings of the International Workshop on Combinatorial Image Analysis. In \underline{Tibor Lukić}, \underline{Reneta Barneva}, \underline{Valentin Brimkov} et al. (ed.): Lecture Notes in Computer Science. Springer, Cham.
\end{hangparas}
%
\begin{hangparas}
	{1em}{1} \years{2018 \href{https://osf.io/zp6nv/}{\aiDoi} \href{https://arxiv.org/pdf/1805.09108}{\aiarXiv}} \textbf{Luciano Melodia}: Deep Learning Estimation of Absorbed Radiation Dose for Nuclear Medicine Diagnostics. Library of the \underline{University of Regensburg}, \underline{Master Thesis} in Information Science.
\end{hangparas}
%
\begin{hangparas}
	{1em}{1} \years{2015 \href{https://www.logos-verlag.de/cgi-bin/engbuchmid?isbn=3808&lng=deu}{\aiDoi} \href{https://ling.auf.net/lingbuzz/004798}{\faFilePdfO}} \textbf{Luciano Melodia}: On the Use of the Paradigm \emph{brauchen} with and without \emph{zu} with Infinitives. In \underline{Kateřina Šichová}, \underline{Reinhard Krapp}, \underline{Paul Rössler} et al. (ed.): Standard Varieties of German -- Case Studies from Social Practice, Logos, Berlin.
\end{hangparas}
%
\vspace{0.5cm}

\newpage
%%%%%%%%%%%%%%%%%%%%%%%%%%%%%%%%%%%%%%%%%%%%%%%%%%%%%%%%%%%%%%%%%%%%%%%%
\section*{Education}
%%%%%%%%%%%%%%%%%%%%%%%%%%%%%%%%%%%%%%%%%%%%%%%%%%%%%%%%%%%%%%%%%%%%%%%%
\years{2024 -- 26, M.Sc.} Mathematics, Friedrich-Alexander University Erlangen-Nürnberg.\\
Minor: Digital Humanities. \\
\years{2021 -- 24, B.Sc.} Mathematics, Friedrich-Alexander University Erlangen-Nürnberg.\\
Topic: Algebraic and Topological Persistence. \\ Minor: Computer Science. \\ 
\years{2015 -- 18, M.A.} Information Science, Regensburg University. \\ 
Topic: Deep Learning for Radiation Dose Calculation.\\ 
\years{2012 -- 15, B.A.} German Philology, Regensburg University\\
Topic: Information Retrieval and Punctuation.\\ 
Majors: Italian Philology, Information Science, Media Informatics. \\ 
\years{2012 -- 13} Web Developer, Rechenzentrum Regensburg University. \\ 
\years{2012} Abitur, Albertus-Magnus-Gymnasium, Regensburg.

%%%%%%%%%%%%%%%%%%%%%%%%%%%%%%%%%%%%%%%%%%%%%%%%%%%%%%%%%%%%%%%%%%%%%%%%
\section*{Certificates}
%%%%%%%%%%%%%%%%%%%%%%%%%%%%%%%%%%%%%%%%%%%%%%%%%%%%%%%%%%%%%%%%%%%%%%%%
\years{2021} The Rust Programming Language, Udemy. \\ 
\years{2020} The Python Mega Course: Build 10 Real World Applications, Udemy. \\ 
\years{2018} Mathematics for ML - Multivariate Calculus, Imperial College London. \\ 
\years{2018} Mathematics for ML - Linear Algebra, Imperial College London. \\ 
\years{2018} Discrete Mathematics, Shanghai Jiao Tong University. \\ 
\years{2018} Introduction to Complex Analysis, Wesleyan University. \\ 
\years{2018} Exploratory Data Analysis, Coursera. \\ 
\years{2018} Intermediate R - Practice Course, Coursera. \\ 
\years{2018} Intermediate R, Coursera. \\ 
\years{2018} Introduction to R, Coursera. \\ 
\years{2018} Supervised Learning in R - Regression, Coursera. \\ 
\years{2018} Supervised Learning in R - Classification, Coursera. \\ 
\years{2018} Text Mining - Bag of Words, Coursera. \\ 
\years{2018} Deep Learning in Python, Coursera. \\ 
\years{2018} Introduction to Machine Learning, Coursera. \\ 
\years{2018} Intro to Python for Data Science, Coursera. \\ 
\years{2018} Machine Learning Toolbox, Coursera. \\ 
\years{2018} Credit Risk Modeling in R, Coursera. \\ 
\years{2018} Data Visualization in R, Coursera. \\ 
\years{2018} Data Visualization with ggplot2 II, Coursera. \\ 
\years{2018} Data Visualization with ggplot2 I, Coursera. \\ 
\years{2016} Beer Sommelièr, Sperber Bräu.

%%%%%%%%%%%%%%%%%%%%%%%%%%%%%%%%%%%%%%%%%%%%%%%%%%%%%%%%%%%%%%%%%%%%%%%%
\section*{Interests}
%%%%%%%%%%%%%%%%%%%%%%%%%%%%%%%%%%%%%%%%%%%%%%%%%%%%%%%%%%%%%%%%%%%%%%%%
\years{Coding} Python, JavaScript.\\ 
\years{Software} GUDHI, Dionysus, Keras.\\
\years{Languages} German (native), English (C2), Italian (C2), Polish (B2).\\ 
\years{Hobbies} Cooking, Reading.\\ 
\years{Sports} Functional training.

\newpage
%%%%%%%%%%%%%%%%%%%%%%%%%%%%%%%%%%%%%%%%%%%%%%%%%%%%%%%%%%%%%%%%%%%%%%%%
\section*{References}
%%%%%%%%%%%%%%%%%%%%%%%%%%%%%%%%%%%%%%%%%%%%%%%%%%%%%%%%%%%%%%%%%%%%%%%%
\years{Prof. Ph.D.} Kang Li\\ Department of Mathematics\\ Friedrich-Alexander University Erlangen-Nürnberg\\ Professor for Representation Theory and Operator Algebras\\ 
\years{\hspace{1cm}\faEnvelope} \url{kang.li@fau.de}\\ 
\years{\hspace{1cm}\faPhone} +49 9131 85-67060\\ 
	
\years{Prof. Dr.-Ing.} Richard Lenz\\ Department of Computer Science\\ Friedrich-Alexander University Erlangen-Nürnberg\\ Professor for Evolutionary Data Management\\ 
\years{\hspace{1cm}\faEnvelope} \url{richard.lenz@fau.de}\\ 
\years{\hspace{1cm}\faPhone} +49 9131 85-27899\\ 
	
\years{Prof. Dr. rer. nat.} Elmar Lang\\ Department of Biophysics\\ Professor for Computational Intelligence\\ 
\years{\hspace{1cm}\faEnvelope} \url{elmar.w.lang@ur.de}\\ 
	
\years{Prof. Dr. phil.} Paul Rössler\\ Department of German Philology\\ Professor for German Linguistics\\ 
\years{\hspace{1cm}\faEnvelope} \url{paul.roessler@ur.de}\\ 
\years{\hspace{1cm}\faPhone} +49 941 943-3444\\
\end{document}