\documentclass[10pt, letterpaper]{article}

\usepackage[
    ignoreheadfoot, 
    top=2 cm, 
    bottom=2 cm, 
    left=2 cm, 
    right=2 cm, 
    footskip=1.0 cm, 
    ]{geometry} 
\usepackage{titlesec} 
\usepackage{tabularx} 
\usepackage{array} 
\usepackage[dvipsnames]{xcolor} 
\definecolor{primaryColor}{RGB}{0, 79, 144} 
\usepackage{enumitem} 
\usepackage{fontawesome5} 
\usepackage{amsmath, amsfonts, amssymb, amsthm} 
\usepackage[
    pdftitle={Luciano Melodias Curriculum Vitae},
    pdfauthor={Luciano Melodia},
    pdfcreator={Luciano Melodia},
    colorlinks=false,
    hidelinks
]{hyperref} 
\usepackage[pscoord]{eso-pic} 
\usepackage{calc} 
\usepackage{bookmark} 
\usepackage{lastpage} 
\usepackage{changepage} 
\usepackage{paracol} 
\usepackage{ifthen} 
\usepackage{needspace} 
\usepackage{iftex} 

\ifPDFTeX
    \input{glyphtounicode}
    \pdfgentounicode=1
    
    \usepackage[utf8]{inputenc}
    \usepackage{lmodern}
\fi



\AtBeginEnvironment{adjustwidth}{\partopsep0pt} 
\pagestyle{empty} 
\setcounter{secnumdepth}{0} 
\setlength{\parindent}{0pt} 
\setlength{\topskip}{0pt} 
\setlength{\columnsep}{0cm} 
\makeatletter
\let\ps@customFooterStyle\ps@plain 
\patchcmd{\ps@customFooterStyle}{\thepage}{
    \color{gray}\textit{\small Luciano Melodia -- Curriculum Vitae \thepage{} of \pageref*{LastPage}}
}{}{} 
\makeatother
\pagestyle{customFooterStyle}

\titleformat{\section}{\needspace{4\baselineskip}\bfseries\large}{}{0pt}{}[\vspace{1pt}\titlerule]

\titlespacing{\section}{-1pt}{0.3 cm}{0.2 cm} 

\renewcommand\labelitemi{$\circ$} 
\newenvironment{highlights}{
    \begin{itemize}[
        topsep=0.10 cm,
        parsep=0.10 cm,
        partopsep=0pt,
        itemsep=0pt,
        leftmargin=0.4 cm + 10pt
    ]
}{
    \end{itemize}
} 

\newenvironment{highlightsforbulletentries}{
    \begin{itemize}[
        topsep=0.10 cm,
        parsep=0.10 cm,
        partopsep=0pt,
        itemsep=0pt,
        leftmargin=10pt
    ]
}{
    \end{itemize}
} 


\newenvironment{onecolentry}{
    \begin{adjustwidth}{
        0.2 cm + 0.00001 cm
    }{
        0.2 cm + 0.00001 cm
    }
}{
    \end{adjustwidth}
} 

\newenvironment{twocolentry}[2][]{
    \onecolentry
    \def\secondColumn{#2}
    \setcolumnwidth{\fill, 4.5 cm}
    \begin{paracol}{2}
}{
    \switchcolumn \raggedleft \secondColumn
    \end{paracol}
    \endonecolentry
} 

\newenvironment{header}{
    \setlength{\topsep}{0pt}\par\kern\topsep\centering\linespread{1.5}
}{
    \par\kern\topsep
} 

\newcommand{\placelastupdatedtext}{
  \AddToShipoutPictureFG*{
    \put(
        \LenToUnit{\paperwidth-2 cm-0.2 cm+0.05cm},
        \LenToUnit{\paperheight-1.0 cm}
    ){\vtop{{\null}\makebox[0pt][c]{
        \small\color{gray}\textit{Last updated in \today}\hspace{\widthof{Last updated in September 2024}}
    }}}
  }
}

\let\hrefWithoutArrow\href

\renewcommand{\href}[2]{\hrefWithoutArrow{#1}{\ifthenelse{\equal{#2}{}}{ }{#2 }\raisebox{.15ex}{\footnotesize \faExternalLink*}}}


\begin{document}
\newcommand{\AND}{\unskip
	\cleaders\copy\ANDbox\hskip\wd\ANDbox
	\ignorespaces
}
\newsavebox\ANDbox
\sbox\ANDbox{}

\placelastupdatedtext
\begin{header}
	\textbf{\fontsize{24 pt}{24 pt}\selectfont Luciano Melodia}
	
	\vspace{0.3 cm}
	
	\normalsize
	\mbox{{\color{black}\footnotesize\faMapMarker*}\hspace*{0.13cm}Erlangen}
	\kern 0.25 cm
	\AND
	\kern 0.25 cm
	\mbox{\hrefWithoutArrow{mailto:melodia.luciano@proton.me}{\color{black}{\footnotesize\faEnvelope[regular]}\hspace*{0.13cm}melodia.luciano@proton.me}}
	\kern 0.25 cm
	\AND
	\kern 0.25 cm
	\mbox{\hrefWithoutArrow{tel:+49 175 3372526}{\color{black}{\footnotesize\faPhone*}\hspace*{0.13cm}+49 175 3372526}}
	\kern 0.25 cm
	\AND
	\kern 0.25 cm
	\AND
	\kern 0.25 cm
	\mbox{\hrefWithoutArrow{https://github.com/karhunenloeve}{\color{black}{\footnotesize\faGithub}\hspace*{0.13cm}karhunenloeve}}
\end{header}

\vspace{0.3 cm - 0.3 cm}



\section{Education}  
\begin{twocolentry}{          
	\textit{Oct 2024 – March 2026}}
	\textbf{Friedrich-Alexander Universität Erlangen-Nürnberg} \\
	\textit{M.Sc. Mathematics, Digital Humanities}
\end{twocolentry}
\vspace{0.10 cm}
\begin{onecolentry}
	\begin{highlights}
		\item \textbf{Thesis:} Universal Coefficients for Homology of Ample Groupoids
		\item \textbf{Topics:} Algebraic Topology, Tensor Categories, Homological Algebra
	\end{highlights}
\end{onecolentry}

\vspace{0.10 cm}
\begin{twocolentry}{           
	\textit{Oct 2021 – Sept 2024}}
	\textbf{Friedrich-Alexander Universität Erlangen-Nürnberg} \\
	\textit{B.Sc. Mathematics, Computer Science}
\end{twocolentry}
\vspace{0.10 cm}
\begin{onecolentry}
	\begin{highlights}
		\item \textbf{Thesis:} Algebraic and Topological Persistence
		\item \textbf{Topics:} Applied Topology, Persistent Homology
	\end{highlights}
\end{onecolentry}

\vspace{0.10 cm}
\begin{twocolentry}{           
	\textit{April 2015 – March 2018}}
	\textbf{Universität Regensburg} \\
	\textit{M.A. Information Science}
\end{twocolentry}
\vspace{0.10 cm}
\begin{onecolentry}
	\begin{highlights}
		\item \textbf{Thesis:} Deep Learning Estimation of Absorbed Radiation Dose for Nuclear Medicine Diagnostics
		\item \textbf{Topics:} Machine Learning, Deep Learning, Matrix Factorization
	\end{highlights}
\end{onecolentry}

\vspace{0.10 cm}
\begin{twocolentry}{           
	\textit{Oct 2012 – March 2015}}
	\textbf{Universität Regensburg} \\
	\textit{B.A. German Philology, Italian Philology, Media Informatics, Information Science}
\end{twocolentry}
\vspace{0.10 cm}
\begin{onecolentry}
	\begin{highlights}
		\item \textbf{Thesis:} Development of a Punctuation Platform with Linguistic Modules for Information Retrieval
		\item \textbf{Topics:} Natural Language Processing, Information Retrieval, Electronic Corpora
	\end{highlights}
\end{onecolentry}

\vspace{0.10 cm}
\begin{twocolentry}{           
	\textit{Oct 2012 – April 2013}}
	\textbf{Universität Regensburg} \\
	\textit{Studienbegleitende IT-Ausbildung}
\end{twocolentry}

\vspace{0.10 cm}
\begin{twocolentry}{           
	\textit{Sept 2004 - May 2012}}
	\textbf{Albertus-Magnus-Gymnasium Regensburg} \\
	\textit{Abitur}
\end{twocolentry}



    
\section{Experience}
\begin{twocolentry}{
    \textit{FAU Erlangen-Nürnberg}    
    
    \textit{April 2023 – March 2026}}
    \textbf{Student Assistant}
    
    \textit{Algebra and Geometry \\ Applied Analysis \\ Applied Mathematics \\ Representation Theory and Operator Algebras \\ Representation Theory and Mathematical Physics}
\end{twocolentry}

\vspace{0.10 cm}
\begin{onecolentry}
	\begin{highlights}
        \item 2026, 2023 Tutor in Topology
        \item 2026, 2024 Tutor in Analysis 3
		\item 2025 Tutor in Mathematics for Engineers A4 (stochastic)
        \item 2025 Tutor in Analysis 2
        \item 2024 Tutor in Topology and Applications
        \item 2024 Tutor in Linear Algebra 1
        \item Lecture on the Tietze Extension Theorem
        \item Lecturer for mathematical proof sessions
        \item Lecturer for exercise sessions
        \item Supervision and correction of written exams
	\end{highlights}
\end{onecolentry}
\vspace{0.2 cm}

\begin{twocolentry}{
    \textit{Erlangen}    
    
    \textit{Jan 2024 – Dez 2025}}
    \textbf{Tutor}
    
    \textit{Private}
\end{twocolentry}

\vspace{0.10 cm}
\begin{onecolentry}
    Preparation for
	\begin{highlights}
		\item mathematics, 2 students, undergraduate studies
		\item computer science, 1 student, undergraduate studies
        \item chemical and biological engineering, 1 student, undergraduate studies
        \item physics, 1 student, undergraduate studies
        \item Abitur, 13 students, bavarian Gymnasium
        \item secondary school, 2 students, bavarian Mittelschule
        \item secondary school, 3 students, bavarian Realschule
        \item elementary school, 1 student, bavarian Grundschule
	\end{highlights}
\end{onecolentry}
\vspace{0.2 cm}

\begin{twocolentry}{
    \textit{Erlangen}    
    
    \textit{Aug 2021 – Aug 2022}}
    \textbf{Werkstudent}
    
    \textit{Corscience GmbH \& Co. KG}
\end{twocolentry}

\vspace{0.10 cm}
\begin{onecolentry}
	\begin{highlights}
		\item Deep convolutional networks trained on multiple GPUs for automatic detection of calibration spikes in ECG data; achieved an accuracy of over ninetynine percent on ten-fold cross validation with a data set of about one million real world samples tested with sigma five significance, which is state of the art
        \item Residual networks for detection of ECG curves in documents; achieved an IOU of approximately ninetyeight percent on ten-fold cross validation with a data set of about ten million artificially enlarged samples using generative neural networks tested with sigma three significance, which is state of the art
        \item Image segmentation using matrix factorisation techniques to isolate ECG curves. Achieved an IOU of approximately ninetynine percent tested with sigma six significance, which is state of the art
	\end{highlights}
\end{onecolentry}
\vspace{0.2 cm}

\begin{twocolentry}{
    \textit{FAU Erlangen-Nürnberg}    
    
    \textit{Sept 2018 – Dez 2021}}
    \textbf{Staff Applied Research Scientist}
    
    \textit{Siemens Energy AG \\ Chair for Evolutionary Data Management}
\end{twocolentry}

\vspace{0.10 cm}
\begin{onecolentry}
	\begin{highlights}
		\item Development of a novel interpolation technique based on the topology of power plant signals to augment reliably data for large scale machine learning training. Validation was performed using theoretical guarantees and practical tests involving handwriting and signals from four different power plants. This novel method has been published at the International Workshop on Combinatorial Image Analysis. \href{https://github.com/karhunenloeve/SIML}{The code is open source: Package SIML} \faGithub.
		\item A novel technique has been developed to estimate roughly the number of parameters in hidden-layer neural networks for signal data. The results were presented at the International Conference on Pattern Recognition. These results have been validated using tailored toy datasets and real power plant signals from a total of four different power plants. Therefore, the technique can be scaled up for big data. Cost savings from reducing training time and current have been estimated to 25.000€ per year. The neural networks achieved close-to-perfect accuracy in classifying the signal data. \href{https://github.com/karhunenloeve/NTOPL}{The code is open source: Package NTOPL} \faGithub.
		\item These results were then used to train a classifier for the KKS (Kraftwerkskennzeichensystem). The aim was to classify each stage of the power plant's hierarchical labelling system down to the level of a sensor. First, the data was enriched using reliable Voronoi interpolation. We then used this estimate to create a trainable, tractable neural network. Finally, we designed a neural network that takes preprocessed signals and their Betti zero and Betti one curves as input. We achieved an accuracy of over 93\% for the coarsest KKS stage and up to 54\% for the physical entity using data from four power plants over multiple years. These results have been published in the proceedings of the European Conference on Principles and Practice of Knowledge Discovery in Databases. \href{https://github.com/karhunenloeve/TwirlFlake}{The code is open source: Package Twirlflake} \faGithub.
		\item Correction of written exams
        \item Assistance in oral exams
        \item Preparation and execution of eletronic exams
        \item Participation in the data science program
        \item Supervision and execution of
        \begin{enumerate}
            \item 2021 lecture on Knowledge Discovery in Databases (evaluation $\varnothing 1.56$)
            \item 2020 seminar on Persistent Homology in Data Analytics
            \item 2020 seminar on Topological Data Analysis (evaluation $\varnothing 1.14$)
            \item 2019, 2020, 2021 exercises in Process Oriented Information Systems (evaluation $\varnothing 1.18$)
            \item 2018, 2019, 2020, 2021 seminar on New Technologies in Data Management
            \item 2018, 2019, 2020, 2021 exercises in Computer Science for Engineers
            \item 2018 exercises in Conceptual Modeling
        \end{enumerate}
        \item Supervision of theses:
        \begin{enumerate}
            \item M.Sc. Sauerhammer (2021): A Classification Dashboard for Sensor Signals from Power Plants
            \item M.Sc. Seidel (2020): Classification of Microbes using Time Series Gas Sensor Array Data
            \item M.Sc. Siddiqui (2020): Extraction of Fetal and Maternal Heart-beats from ECG Signals
            \item B.Sc. Hahn (2021): Classification of Sensor Signals from Power Plants
            \item B.Sc. Schäfer (2021): Learning Validation Models from Sensors of a
Power Plant
        \end{enumerate}
        \item Programming with CUDA v.11.0, Tensorflow 2.4, CuDNN v.8.0.4. in Python v.3.8 and v.3.9
        \item Operating systems: Ubuntu 20.04, Solus 4, Archlinux 5.11, Windows 11
	\end{highlights}
\end{onecolentry}
\vspace{0.2 cm}

\begin{twocolentry}{
    \textit{Regensburg}    
    
    \textit{June 2015 – March 2018}}
    \textbf{Data Scientist}
    
    \textit{mb Support GmbH}
\end{twocolentry}

\vspace{0.10 cm}
\begin{onecolentry}
	\begin{highlights}
		\item Industrial document-digitization pipeline for mass paper-pile scanning; scanning-street engineering; one-percent character-error-rate OCR via Google Cloud Vision and custom recurrent neural networks; ergonomic user-interface integration for Openviva C2; Industrial scanning throughput benchmark: capacity for approximately sixty million documents in continuous operation
        \item Asterisk telecommunication-API integration into Openviva C2 with ergonomic design, roughly five thousand lines of PL/SQL and Python
        \item Statistical data and market analysis with deep neural and convolutional networks and regression methods
	\end{highlights}
\end{onecolentry}
\vspace{0.2 cm}

\begin{twocolentry}{
    \textit{Universität Regensburg}    
    
    \textit{Oct 2013 – Sept 2015}}
    \textbf{Research Assistant}
    
    \textit{Chair for German Linguistics}
\end{twocolentry}

\vspace{0.10 cm}
\begin{onecolentry}
	\begin{highlights}
		\item Proofreading of books and papers
        \item Correction of exams
        \item Organisation of conferences
        \item Maintenance of the university website
        \item Implementation of a scientific social network featuring a custom JavaScript-written search engine for real-time online usage, comprising approximately a thousand lines of code
	\end{highlights}
\end{onecolentry}
\vspace{0.2 cm}

\begin{twocolentry}{
    \textit{Regensburg}    
    
    \textit{Sept 2012 – Dez 2015}}
    \textbf{Chef}
    
    \textit{Apostelkeller}
\end{twocolentry}

\vspace{0.10 cm}
\begin{onecolentry}
	\begin{highlights}
		\item Cooking with menu for up to 140 guests
        \item Waitressing
        \item Stock management
	\end{highlights}
\end{onecolentry}

\vspace{0.2 cm}
\begin{twocolentry}{
    \textit{Regensburg}    
    
    \textit{Nov 2012 – Mai 2015}}
    \textbf{Service Staff}
    
    \textit{Trademarketing Service GmbH}
\end{twocolentry}

\vspace{0.10 cm}
\begin{onecolentry}
	\begin{highlights}
		\item Goods management, receipt and ordering.
	\end{highlights}
\end{onecolentry}


\vspace{0.2 cm}
\begin{twocolentry}{
    \textit{Regensburg}    
    
    \textit{Oct 2012 – Aug 2014}}
    \textbf{Translator}
    
    \textit{Anatol GmbH \& Co. KG}
\end{twocolentry}

\vspace{0.10 cm}
\begin{onecolentry}
	\begin{highlights}
		\item Translation between Italian -- German -- Polish -- English
	\end{highlights}
\end{onecolentry}

\vspace{0.2 cm}
\begin{twocolentry}{
    \textit{Regensburg}
    
    \textit{Aug 2010}}
    \textbf{Volunteer}
    
    \textit{Alten- und Pflegeheim St. Josef}
\end{twocolentry}
\vspace{0.10 cm}

\section{Publications}
\begin{samepage}
	\begin{twocolentry}{2021}
		\textbf{Homological Time Series Analysis of Sensor Signals from Power Plants.}
		
        \mbox{\textbf{\textit{Luciano Melodia}}}, \mbox{Richard Lenz}
	\end{twocolentry}
	\begin{onecolentry}
		\href{https://doi.org/10.1007/978-3-030-93736-2\_22}{10.1007/978-3-030-93736-2\_22}
	\end{onecolentry}
	\vspace{0.10 cm}
	\begin{twocolentry}{2021}
		\textbf{Estimate of the Neural Network Dimension Using Algebraic Topology and Lie Theory.}
		
        \mbox{\textbf{\textit{Luciano Melodia}}}, \mbox{Richard Lenz}
	\end{twocolentry}
	\begin{onecolentry}
		\href{https://doi.org/10.1007/978-3-030-68821-9\_2}{10.1007/978-3-030-68821-9\_2}
	\end{onecolentry}
	\vspace{0.10 cm}
	\begin{twocolentry}{2020}
		\textbf{Persistent Homology as a Stopping Criterion for Voronoi Interpolation.}
		
        \mbox{\textbf{\textit{Luciano Melodia}}}, \mbox{Richard Lenz}
	\end{twocolentry}
	\begin{onecolentry}
		\href{https://doi.org/10.1007/978-3-030-51002-2\_3}{10.1007/978-3-030-51002-2\_3}
	\end{onecolentry}
	\vspace{0.10 cm}
	\begin{twocolentry}{2015}
		\textbf{Zur Verwendung des Paradigmas \textit{brauchen} mit und ohne \textit{zu} mit Infinitiv.}
		
        \mbox{\textbf{\textit{Luciano Melodia}}}
	\end{twocolentry}
	\begin{onecolentry}
		\href{https://www.logos-verlag.de/cgi-bin/engbuchmid?isbn=3808&lng=eng&id=}{ISBN 978-3-8325-3808-8}
	\end{onecolentry}
\end{samepage}


    
\section{Conferences}        
\begin{twocolentry}{\textit{2022-24}}\href{https://logconference.org/program-committee/}{Learning on Graphs, LOG}\end{twocolentry}
\begin{onecolentry}
	\begin{highlights}
		\item Reviewer, Program Committee
	\end{highlights}
\end{onecolentry}
\vspace{0.10 cm}
\begin{twocolentry}{\textit{2020-24}}
\href{https://www.iaria.org/conferences/DBKDA.html}{Advances in Databases, Knowledge, and Data Applications} \end{twocolentry}
\begin{onecolentry}
	\begin{highlights}
		\item Reviewer, Program Committee
	\end{highlights}
\end{onecolentry}
\vspace{0.10 cm}
\begin{twocolentry}{\textit{2022}}
\href{https://gt-rl.github.io/}{Geometrical and Topological Representation Learning, Workshop at ICLR} \end{twocolentry}
\begin{onecolentry}
	\begin{highlights}
		\item Reviewer, Program Committee
	\end{highlights}
\end{onecolentry}
\vspace{0.10 cm}
\begin{twocolentry}{\textit{2021}}
\href{https://link.springer.com/conference/icpr}{International Conference on Pattern Recognition} \end{twocolentry}
\begin{onecolentry}
	\begin{highlights}
		\item Author, full paper
	\end{highlights}
\end{onecolentry}
\vspace{0.10 cm}
\begin{twocolentry}{\textit{2020}}
\href{https://iwcia2020.wordpress.com/}{International Workshop on Combinatorial Image Analysis} \end{twocolentry}
\begin{onecolentry}
	\begin{highlights}
		\item Author, full paper
	\end{highlights}
\end{onecolentry}
\vspace{0.10 cm}
\begin{twocolentry}{\textit{2020}}
\href{https://cnc.ac.in/assets/uploads/Faculty/profile/Mr.\%20Rajeshkumar\%20Resume\_.pdf}{International Conference on Practical Mathematical Discourse} \end{twocolentry}
\begin{onecolentry}
	\begin{highlights}
		\item Guest talk, Introduction to Persistent Homology
	\end{highlights}
\end{onecolentry}
\vspace{0.10 cm}
\begin{twocolentry}{\textit{2020}}
\href{https://tda-in-ml.github.io/committee}{Topological Data Analysis and Beyond, Workshop at NeuRIPS} \end{twocolentry}
\begin{onecolentry}
	\begin{highlights}
		\item Reviewer, Program Committee
	\end{highlights}
\end{onecolentry}
\vspace{0.10 cm}
\begin{twocolentry}{\textit{2020}}
\href{https://sigmod2020.org/}{Symposion on Principles of Database Systems, SIGMOD/PODS} \end{twocolentry}
\vspace{0.10 cm}
\begin{twocolentry}{\textit{2019-2020}}
\href{https://ecmlpkdd2019.org/}{European Conference on Machine Learning and Principles and Practice of Knowledge Discovery in Databases, ECML/PKDD} \end{twocolentry}
\begin{onecolentry}
	\begin{highlights}
		\item Author, full paper
	\end{highlights}
\end{onecolentry}
\vspace{0.10 cm}
\begin{twocolentry}{\textit{2015}}
\href{https://www.uni-regensburg.de/sprache-literatur-kultur/germanistik/internationales/forschung-und-lehre/index.html}{Sprachmanagement \& Orthografie} \end{twocolentry}
\vspace{0.10 cm}
\begin{twocolentry}{\textit{2013}}
\href{https://www.uni-regensburg.de/sprache-literatur-kultur/germanistik/internationales/forschung-und-lehre/index.html}{Destandardisierung und Standardvarietät} \end{twocolentry}
\begin{onecolentry}
	\begin{highlights}
		\item Author, full paper
	\end{highlights}
\end{onecolentry}


\section{Awards, Grants and Service}        
\begin{twocolentry}{\textit{2024}}
\href{https://logconference.org/program-committee/}{Top Reviewer Award from Learning on Graphs} \end{twocolentry}
\vspace{0.10 cm}
\begin{twocolentry}{\textit{2024}}
\href{https://www.stmwk.bayern.de/studenten/foerderung-und-stipendien/forster-stipendium.html}{Oskar-Karl-Forster Scholarship Fellow} \end{twocolentry}
\vspace{0.10 cm}
\begin{twocolentry}{\textit{2024}}
Student Representative for the Department of Mathematics at Friedrich-Alexander Universität Erlangen-Nürnberg\end{twocolentry}
\vspace{0.10 cm}
\begin{twocolentry}{\textit{2019-20}}
\href{https://gi.de/}{Member of the Gesellschaft für Informatik e.V.}\end{twocolentry}
\vspace{0.10 cm}
\begin{twocolentry}{\textit{2017-18}}
\href{https://www.uni-regensburg.de/informatik-data-science/maschinelles-lernen-insb-uncertainty-quantification/startseite/index.html}{Member of the Computational Intelligence and Machine Learning Group}\end{twocolentry}
\vspace{0.10 cm}
\begin{twocolentry}{\textit{2016}}
Student Representative for the Department of Language, Literature and Cultural Sciences at Universität Regensburg\end{twocolentry}

\section{Addendum}  
\begin{onecolentry}
	\textbf{Programming:} Python, Rust, C++
\end{onecolentry}
\vspace{0.2 cm}
\begin{onecolentry}
	\textbf{Web Technologies:} HTML5, CSS3, Javascript, PHP
\end{onecolentry}
\vspace{0.2 cm}
\begin{onecolentry}
	\textbf{Typesetting:} \LaTeX
\end{onecolentry}
\vspace{0.2 cm}
\begin{onecolentry}
	\textbf{Operating Systems:} Arch Linux, Ubuntu, Mac OS, Windows
\end{onecolentry}
\vspace{0.2 cm}
\begin{onecolentry}
	\textbf{Languages:} German native, English C2, Italian C2, Polish B2, Spanish A2
\end{onecolentry}
\vspace{0.2 cm}
\begin{onecolentry}
	\textbf{Sports:} Boxing, Muay Thai, Weng Chun, Table Tennis
\end{onecolentry}
\vspace{0.2 cm}
\begin{onecolentry}
	\textbf{Hobbies:} Cooking, Novels
\end{onecolentry}

\newpage
\section{References}
\begin{twocolentry}{\textit{2025-26}}\href{https://www.math.fau.de/lie-gruppen/personen/karl-hermann-neeb/}{Prof. Dr. Karl-Herrmann Neeb}\end{twocolentry}
\begin{onecolentry}
	\begin{highlights}
		\item Department of Mathematics
        \item Friedrich-Alexander Universität Erlangen-Nürnberg
        \item Professor for Lie Groups and Representation Theory
        \item[\faEnvelope] \url{neeb@math.fau.de}
        \item[\faPhone] +49 9131 85-67037
	\end{highlights}
\end{onecolentry}
\vspace{0.2cm}
\begin{twocolentry}{\textit{2024-26}}\href{https://www.math.fau.de/lie-gruppen/personen/catherine-meusburger/}{Prof. Dr. Catherine Meusburger}\end{twocolentry}
\begin{onecolentry}
	\begin{highlights}
		\item Department of Mathematics
        \item Friedrich-Alexander Universität Erlangen-Nürnberg
        \item Professor for Representation Theory and Mathematical Physics
        \item[\faEnvelope] \url{catherine.meusburger@fau.de}
        \item[\faPhone] +49 9131 85-67034
	\end{highlights}
\end{onecolentry}
\vspace{0.2cm}
\begin{twocolentry}{\textit{2022-26}}\href{https://sites.google.com/site/kanglishomepage/}{Prof. Dr. Kang Li}\end{twocolentry}
\begin{onecolentry}
	\begin{highlights}
		\item Department of Mathematics
        \item Friedrich-Alexander Universität Erlangen-Nürnberg
        \item Professor for Representation Theory and Operator Algebras
        \item[\faEnvelope] \url{kang.li@fau.de}
        \item[\faPhone] +49 9131 85-67060
	\end{highlights}
\end{onecolentry}
\vspace{0.2cm}
\begin{twocolentry}{\textit{2018-21}}\href{https://sites.google.com/site/kanglishomepage/}{Prof. Dr. Richard Lenz}\end{twocolentry}
\begin{onecolentry}
	\begin{highlights}
		\item Department of Computer Science
        \item Friedrich-Alexander Universität Erlangen-Nürnberg
        \item Professor for Evolutionary Data Management
        \item[\faEnvelope] \url{richard.lenz@fau.de}
        \item[\faPhone] +49 9131 85-27899
	\end{highlights}
\end{onecolentry}
\vspace{0.2cm}
\begin{twocolentry}{\textit{2018-21}}\href{https://epub.uni-regensburg.de/view/people/Lang=3AElmar_Wolfgang=3A=3A.default.html}{Prof. Dr. em. Elmar Lang}\end{twocolentry}
\begin{onecolentry}
	\begin{highlights}
		\item Department of Biophysics
        \item Universität Regensburg
        \item Professor for Computational Intelligence
        \item[\faEnvelope] \url{elmar.w.lang@ur.de}
	\end{highlights}
\end{onecolentry}
\vspace{0.2cm}
\begin{twocolentry}{\textit{2013-16}}\href{https://www.uni-regensburg.de/sprache-literatur-kultur/germanistik-sw-1/roessler/index.html}{Prof. Dr. Paul Rössler}\end{twocolentry}
\begin{onecolentry}
	\begin{highlights}
		\item Department of German Philology
        \item Universität Regensburg
        \item Professor for German Linguistics
        \item[\faEnvelope] \url{paul.roessler@ur.de}
        \item[\faPhone] +49 941 943-3444
	\end{highlights}
\end{onecolentry}
\end{document}